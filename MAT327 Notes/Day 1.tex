\section{Day 1: Open Sets and Continuity (Sep. 3, 2024)}
\begin{abstract}
    \sffamily\small
    \noindent This class is \textit{MAT327}; $3$ meaning third year, $2$ meaning the contents are on the fundamental side, and $7$ meaning no mercy.

    -- Dror Bar-Natan
\end{abstract}

Course administration matters first;
\begin{itemize}
    \item The course link is given \href{https://drorbn.net/24-327}{here} (this will link straight to Quercus).
    \item The textbook is \href{https://www.pearson.com/en-ca/subject-catalog/p/topology-classic-version/P200000006299/9780137848669}{James Munkres' Topology} (online PDF: \href{https://people.math.ethz.ch/~dkosanovic/24-FS/Munkres-Topology.pdf}{ETH Zurich mirror}); Prof Bar-Natan strongly recommends a paper copy, though (since people get distracted on the computer).
\end{itemize}

\hrule \bigskip

\noindent Today's reading in the textbook is on Ch. 1, sections 1 to 8, and Ch. 2, sections 12 to 13. Readings are supplementary to lecture material\footnote{iirc it won't be tested unless specified. its still good to learn tho}. The goal of this course is to understand continuity in its most general form; in particular,
\begin{itemize}
    \item In MAT157, we studied continuity in $f : \RR \to \RR$;
    \item In MAT257, we will study continuity in $f : \RR^n \to \RR^m$;
\end{itemize} 
but in this class, we will study continuity in $F : X \to Y$, where $X, Y$ are arbitrary spaces, such as (but not limited to) $\RR^n$, $\RR^\NN$, $\{0, 1\}^\NN$ (binary sequences), and so on. We start with some refreshers on previous coursework;

\begin{definition}[Continuity in $\RR^n \to \RR^m$]
    A function $f : \RR^n \to \RR^m$ is called ``continuous'' if it is continuous at all points in $\RR^n$. Specifically, for all $x_0 \in \RR^n$ and $\eps > 0$, there exists $\delta > 0$ such that $\abs{x - x_0} < \delta \implies \abs{f(x) - f(x_0)} < \eps$.
\end{definition}

\noindent With this, we state our main theorem for today (proof given later),

\begin{simplethm}[Continuity on $\RR^n$ if and only pre-image of open subsets is open]
    A function $f : \RR^n \to \RR^m$ is continuous if and only if all open subsets $U \in \RR^m$ have $f^{-1}(U)$ open.
\end{simplethm}

\noindent In order to build up to the above, we start by defining some terms;

\begin{definition}[Open Ball]
    Let $r > 0$, and $x \in \RR^n$. An open ball of radius $r$ about $x$ is given formally by $B_r(x) = \{y \in \RR^n : \abs{x - y} < r\}$. Visually, we have
    \begin{center} \begin{tikzpicture}[scale=0.75]
        \draw[->] (-2, 0) -- (8, 0);
        \draw[->] (0, -1.5) -- (0, 5);
        \node at (0, 0) {\textbullet} ++(-0.3, -0.3) node {$O$}; % new technique LOL
        
        \draw[dashed] (3.7, 2.5) circle (1.8);
        \draw (3.7, 2.5) -- node[above] {$r$} (5.15, 3.567);
        \node[crimson] at (3.7, 2.5) {\textbullet} ++(3.4, 2.2) node {$x$};
    \end{tikzpicture} \end{center}
    Note that the border of the circle is not in $B_r(x)$, as the distance metric asks for a strict inequality.
\end{definition}

\newpage
\noindent In a similar fashion, we have

\begin{definition}[Open Set]
    A set $U \subset \RR^n$ is called \textit{open} if, for all $x \in U$, there exists an open ball about $x$ contained in $U$. Specifically, there is some $\eps > 0$ such that $B_\eps(x) \subset U$.
\end{definition}

\noindent Intuitively, we could say that the set $U$ does not contain its edge; if it did, let $x$ be on said edge; then we would not be able to fit an open ball about $x$ in $U$. Here are some examples of open sets from lecture;
\begin{enumerate}
    \item The whole set $U = \RR^n$ is open.
    \item The empty set $\emptyset$ is open. Since there does not exist any $x \in \emptyset$, no conditions on balls need to be satisfied; ``every dog in the empty set of dogs is green''.
    \item $(0, 1)$ is open on $\RR^1$ (and any open interval, for that matter).
    \item $B_r(x) \in \RR^n$ is open. To see this, observe that for any $y \in B_r(x)$, we may pick $\eps < r - \abs{x - y}$; by triangle inequality, all elements in $B_\eps(y)$ must also be in $B_r(x)$.
\end{enumerate}

\noindent We now define images and pre-images; let us have a function between sets $f : X \to Y$. For subsets $A \subset X$ and $B \subset Y$,
\begin{align*}
    f(A) &= \{f(a) \mid a \in A\}, \\
    f^{-1}(B) &= \{x \in X \mid f(x) \in B\}.
\end{align*}
Since $f$ need not be injective, $f^{-1}$ does not necessarily exist. While images of a union of subsets is a union of the images, the same is not true for intersections; let $A_1, \dots, A_n$ be subsets of $X$; then
\[ f\left( \bigcup_{i=1}^n A_i \right) = \bigcup_{i=1}^n f(A_i), \hspace{0.2in} f\left( \bigcap_{i=1}^n A_i \right) \subset \bigcap_{i=1}^n f(A_i). \]
On the other hand, pre-images preserve both union and intersection; let $B_1, \dots, B_n \subset Y$; then
\[ f^{-1}\left( \bigcup_{i=1}^n A_i \right) = \bigcup_{i=1}^n f^{-1}(A_i), \hspace{0.2in} f^{-1}\left( \bigcap_{i=1}^n A_i \right) = \bigcap_{i=1}^n f^{-1}(A_i). \]
As for set complements, we have
\[ f^{-1}(B^C) = f^{-1}(B)^C, \]
but the same cannot be said for images; one side includes the other depending on injectivity and surjectivity \footnote{check if i'm right on this, i did a quick run thru in my head}.

\medskip
\noindent \textbf{Note:} for the proof of Theorem 1.2, I'll leave it out in case Prof. Bar-Natan proves it later on Thursday.