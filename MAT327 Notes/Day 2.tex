\section{Day 2: Basic Definitions and Topological Spaces (Sep. 5, 2024)}
We start by recapping the previous lecture; we introduced
\begin{itemize}
    \item Open sets $U \subset \RR^n$, where there exists an open ball of radius $\eps > 0$ about any $x \in U$.
    \item Continuity in $f : \RR^n \to \RR^m$.
\end{itemize}
Recall the continuity property from last class, that $f : \RR^n \to \RR^m$ is continuous if and only if the pre-image of an open set $U$ is open.
\begin{enumerate}
    \item[$(\Rightarrow)$] To start, pick any $x_0 \in f^{-1}(U)$, and let us have a small enough $\eps > 0$ such that $B_\eps(f(x_0)) \in U$ (this is possible since $U$ is open). By continuity, there exists $\delta > 0$ such that any $x \in B_\delta(x_0)$ satisfies $f(x) \in B_\eps(f(x_0)) \subset U$; this means $x \in f^{-1}(U)$ by definition of pre-image, and since $x$ was arbitrary, we see $B_\delta(x_0) \in f^{-1}(U)$. Moreover, since $x_0$ was also arbitrary, our construction shows that there always exists a $\delta$-ball about any point in the pre-image, and so $f^{-1}(U)$ is open. \qed
    \item[$(\Leftarrow)$] For the other direction, take any $x_0 \in \RR^n$, and $\eps > 0$. Since $B_\eps(f(x_0))$ is open we have that $f^{-1}(B_\eps(f(x_0)))$ is open as well. This means we may pick a small enough $\delta > 0$ such that $B_\delta(x) \subset f^{-1}(B_\eps(f(x_0)))$ (by definition of openness), and we immediately see
    \[ f(B_\delta(x_0)) \subset B_\eps(f(x_0)). \]
    This is a reconstruction of the epsilon-delta definition of continuity, and so we are done. \qed 
\end{enumerate}

\noindent Open sets in $\RR^n$ have a number of properties;
\begin{enumerate}[label=(\alph*)]
    \item $\emptyset, \RR^n$ are open sets.
    \item The union of open sets are open; specificay, let $S$ be a set of indices, and let $A_\alpha \subset \RR^n$ for all $\alpha \in S$. Then
    \[ \bigcup_{\alpha \in S} A_\alpha \]
    is an open set.
    \item The finite intersection of open sets are also open. Let $A_1, \dots, A_n \subset \RR^n$ be open; then
    \[ \bigcap_{i=1}^n A_i \]
    is open. 
\end{enumerate}
We now proceed with the proofs for these properties.
\begin{enumerate}[label=(\alph*)]
    \item This was proven last lecture.
    \item For any $x$ in the union, by definition, there exists $\alpha \in S$ such that $x \in A_\alpha$. Then there exists $\eps > 0$ such that $B_\eps(x) \subset A_\alpha$, and $A_\alpha \subset \bigcup_{\alpha \in S} A_\alpha$. This concludes that the union is open.
    \item For any $x$ in the finite intersection, then $x$ is an element of each of $A_1, \dots, A_n$. Let us have $\eps_1, \dots, \eps_n > 0$ such that $B_{\eps_i}(x) \eps A_i$, and take $\eps = \min\{\eps_1, \dots, \eps_n\}$ (note that minimum is defined only for finite lists), which means $B_\eps(x) \subset \bigcap_{i=1}^n A_i$.
\end{enumerate}

\newpage
\noindent In the case of infinite intersection of open sets, the resulting set need not be open. For example, let us consider the infinite intersection of intervals
\[ \bigcap_{j=1}^\infty \left( -\frac{1}{j}, \frac{1}{j} \right) = \{0\}, \] 
which is not open.

\begin{definition}[Topology on a Set]
    Let $X$ be a set equipped with topology $\ST$ on $X$. $\ST$ is a collection of subsets of $X$, i.e. $\ST \subset \SP(X)$, with properties
    \begin{itemize}
        \item $\eps, X \in \ST$.
        \item The union of subsets of $\ST$ is also in $\ST$.
        \item The finite intersection of subsets of $\ST$ is also in $\ST$.
    \end{itemize}
\end{definition}
\noindent We call $(X, \ST)$ a \textit{topological space}, which may be abbreviated to $X$ is $\ST$ is given or obvious. Moreover, we define another notion of openness, where $U \in \ST$ is said to ``be open relative to $\ST$,'' or that ``$U$ is open.'' Here are a few examples of such topological spaces.
\begin{enumerate}[label=(\alph*)]
    \item The standard topology on $\RR^n$ is given by
    \[ \RR_{std} \left(\RR, \ST_{std} = \{ U \subset \RR^n \mid U \text{ is open in the ``old sense''}\}\right), \]
    i.e. collection of open intervals\footnote{this is how i understood it, he might clarify next time?}.
    \item The discrete topology, $X_{discrete} = (X, \ST = \SP(x))$ can be defined over any set $X$, equipped with $\ST$ as the collection of all subsets.
    \item The trivial topology, $X_{trivial} = (X, \ST = \{\emptyset, X\})$.
\end{enumerate}

\begin{definition}[Continuity between Topological Spaces]
    If $(X, \ST_X), (Y, \ST_Y)$ are topological spaces, and $f : X \to Y$ is a function between said spaces, then we say $f$ is continuous if, for all $U \in \ST_Y$, we have that $f^{-1}(U) \in \ST_X$. This draws from our definition from the notion that pre-images of open sets are open.
\end{definition}

\noindent Now for the examples from lecture;
\begin{enumerate}[label=(\alph*)]
    \item $f : X_{discrete} \to \RR_{std}$ is always continuous.
    \item $f : X_{trivial} \to \RR_{std}$ is continuous if and only if $f$ is constant.
    \item $f : \RR_{std} \to X_{trivial}$ is always continuous.
    \item $f : \RR_{std} \to X_{discrete}$ is almost never continuous, except when $X$ is empty or a singleton (in which case, $X_{discrete} = X_{trivial}$). If $x_0 \in X$ yet $X \setminus \{x_0\} \neq \emptyset$, then let us have
    \begin{align*}
        A &:= f^{-1}(\{x_0\}), \\
        B &:= f^{-1}(X \setminus \{x_0\}).
    \end{align*} 
    While $A, B$ are both open, we see that $A \cup B = \RR$, and $A \cap B = \emptyset$.
\end{enumerate}
