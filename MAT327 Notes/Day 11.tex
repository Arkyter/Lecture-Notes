\section{Day 11: Metric Spaces and Products; Quotient Spaces (Oct. 8, 2024)}
Outfit of the day! RAINBOW HAT!!!
\begin{figure}[h]
    \centering
    \includegraphics[scale=0.1]{MAT327 Notes/Dror Shirts/dror day 11 shirt.jpg}
\end{figure}

\noindent Recap of last lecture; suppose $n \in \NN$ and $X_n \neq \emptyset$ is metrizable. Then $X = \prod_n X_n$ is metrizable. Also, given $(X_n, d_n)$ with $d_n$ bounded by $1$ (Lemma 10.5), set $d((x), (y)) = \sup_n \frac{1}{n} d(x_n, y_n)$. Then we claim that
\begin{enumerate}[label=(\alph*)]
    \item $d((x), (y))$ as given above is a metric.
    \item Said metric induces the cylinder topology.
\end{enumerate}
We start by proving (a); i.e., we will check that $\sup_n d_n(x_n, y_n)$ is a metric (i.e., the maximum across all $(X_n, d_n)$). The first two axioms are obvious, so we only check that $d$ satisfies the triangle inequality. Let us have $(x), (y), (z) \in X$, and write
\begin{align*}
    d((x), (y)) + d((y), (z)) &\geq \sup \underbrace{\left(d_n(x_n, y_n)\right)}_{a_n} + \sup \underbrace{\left(d_n(y_n, z_n)\right)}_{b_n} \\
    &= \sup(a_n) + \sup(b_n) \\
    &= \sup \left(d_n(x_n, y_n) + d_n(y_n, z_n)\right) \tag{Triangle Ineq. in $X_n$} \\
    &\geq \sup d_n(x_n, z_n) = d((x), (z)). \qed
\end{align*}

\noindent We now check that (b) holds as well; the proof in class was unclear, so we will simply refer to Theorem 20.5 in Munkres.
\medskip\newline
\noindent We now move onto quotient spaces. To start, we define the notion of equivalence relations. An \textit{equivalence relation} on a set $X$ is a relation $\sim : X \times X \to \{0, 1\}$ (read: true or false) such that it satisfies the following properties,
\begin{enumerate}[label=(\alph*)]
    \item $x \sim x$ (reflexive property);
    \item $x \sim y \implies y \sim x$ (symmetric property);
    \item $x \sim y, y \sim z \implies x \sim z$ (transitive property). 
\end{enumerate}
We now go over a few examples,
\begin{enumerate}[label=(\alph*)]
    \item ``Examples'' $\sim$ ``examples'' under the relation where uppercase and lowercase are conisdered equivalent.
    \item '$=$' is an equivalence relation.
    \item Let $f : X \to Y$. Define $x_1 \sim x_2$ if $f(x_1) = f(x_2)$. Without loss of generality, we may let $f$ be surjective.
\end{enumerate}
\begin{simplethm}
    All equivalence relations come in this way, i.e. equivalence relations on $X$ are equivalent to surjections with domain $X$.
\end{simplethm}
\noindent We start by providing a few more definitions.
\begin{definition}
    Given $X$ be equipped with an equivalence relation $\sim$, i.e. $(X, \sim)$, and let $x_0 \in X$; we have that $X \supset [x_0]$, i.e. $[x_0]$, the equivalence class of $x_0$, which is given by $\{x \in X \mid x_0 \sim x\}$.
\end{definition}
\begin{simplelemma}
    Let us have equivalence classes $[x_0], [x_1]$. Such equivalence classes are either equal or disjoint.
\end{simplelemma}
\noindent Suppose $[x_0] \cap [x_1] \ni z$. Then by definition, $x_0 \sim z \sim x_1$, meaning $x_0 \sim x_1$. Then every element in $[x_0]$ and $[x_1]$ are equivalent to each other by transitivity. Thus, $[x_0] = [x_1]$; if no such $z$ exists, clearly $[x_0]$ and $[x_1]$ are disjoint. \qed
\begin{simplelemma}
    The union of $[x_0]$ over all $x_0 \in X$ is equal to the whole of $X$.
\end{simplelemma}
\begin{definition}
    The set $X/\sim$ is called ``$X$ mod $\sim$''. It is the set of all qeuivalence classes in $X$, $\{[x_0] \mid x_0 \in X\}$. Then let $\pi : X \to X/\sim$, and $\pi : x_0 \mapsto [x_0]$.
\end{definition}
\noindent We now prove the theorem. Given $\sim$ on $X$, $Y := X/\sim$, and $f = \pi$. Then we claim that $\sim_\pi = \sim$. The rest of the proof is skipped.
\medskip\newline
\noindent Suppose $X$ is a topological space, and $\sim$ is an equivalence relation on $X$. We seek a topology on $X/\sim$ such that
\begin{enumerate}[label=(\alph*)]
    \item $\pi : X \to X/\sim$ is continuous.
    \item Given $g : X/\sim \to Z$, if $g \circ \pi$ is continuous, then so is $g$.
\end{enumerate}
\begin{simplethm}
    Such a topology satisfying the above conditions exists, and is unique.
\end{simplethm}
\noindent The proof is left to next class / as an exercise or something.