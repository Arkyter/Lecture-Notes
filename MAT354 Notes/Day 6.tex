\section{Day 6: Cauchy's Theorem on a Disc (Sep. 18, 2025)}
Recall Goursat's theorem from last class, where if $\Omega \subset \CC$ is open and $T \subset \Omega$ is a triangle whose interior is contained in $\Omega$, then for any holomorphic function $f$ on $\Omega$, we have that
\[ \int_T f(z) \, dz = 0. \]
We introduce a follow-up to this thoerem.
\begin{theorem}
    If $f$ is holomorphic on a disc, then $\int_\gamma f(z) \, dz = 0$ for any closed curve $\gamma$ in that disc.
\end{theorem}
\noindent To prove this, we start by using Goursat's theorem to show $f$ has a primitive, and then we complete the proof using the complex FTC, i.e., if $f$ is holomorphic on a disc, then $f$ has a primitive on that disc.
\begin{proof}
    After a translation, we may assume that the center of the disc is $0$. Define $F : D \to \CC$, given by $z \mapsto \int_{\gamma_z} f(u) \, du$. To show that $F$ is holomorphic and $F'(z) = f(z)$, fix $z \in D$, and observe that for any $h \in \CC \setminus \{0\}$ with $z + h \in D$, we have that
    \[ F(z+h) - F(z) = \int_{\gamma_{z+h}} f(u) \, du - \int_{\gamma_z} f(u) \, du. \]
    Regard this as the path from $z$ to $0$ to $z+h$. Let us add to the expression the integrals over two paths, going both directions so that we do not change the value of $F(z+h) - F(z)$, one between $z$ and $\Re(z+h) + i \Im(z)$, and one between $z$ and $z+h$ directly. In this manner, we've created a rectangular region and a triangular region on which we have path integrals over, and per Goursat's theorem, they all vanish, and we are left with the integral on the path $\eta$ from $z$ to $z+h$. This means all that remains is to compute\footnote{GOD KNOWS if this is a $w$ or an $\omega$, i'm just going to use $w$ for now. forensic analysis on yalls handwriting holy shit}
    \[ F(z + h) - F(z) = \int_{\eta} f(w) \, dw \]
    Since $f$ is continuous at $z$, we may write $f(w) = f(z) + \psi(w)$, where $\psi(w) \to 0$ as $w \to z$. This means we may write
    \[ \int_\eta f(w) \, dw = \int_\eta f(z) \, dw + \int_\eta \psi(w) \, dw = f(z)(z+h - z) + \int_\eta \psi(w) \, dw, \]
    upon which we may rearrange and rewrite the above RHS to obtain
    \begin{align*}
        \abs{\frac{F(z+h) - F(z)}{h} - f(z)} &= \abs{\frac{1}{h} \int_\eta \psi(w) \, dw} \\
        &\leq \frac{1}{\abs{h}} \sup_{w \in \eta} \abs{\psi(w)} \underbrace{\length(\eta)}_{=\abs{h}} = \sup_{w \in \eta} \abs{\psi(w)} \taking{h \to 0} 0.
    \end{align*}
    This concludes the hard part of the proof in showing that $f$ has a primitive; by complex FTC, we immediately see that $\int_\gamma f(z) \, dz = 0$, since $\gamma$ is a closed curve and its endpoints are equal to each other.
\end{proof}

\newpage
\noindent We now give an example.
\begin{problem}
    For all $\xi \in \RR$, let $\SF$ denote the Fourier transform, and let
    \[ (\SF f)(\xi) = \int_{-\infty}^\infty f(x) e^{-2\pi i x \xi} \, dx. \]
    Show that if $f(x) = e^{-\pi x^2}$, we have that $(\SF f)(\xi) = f(\xi) = e^{-\pi \xi^2}$.
\end{problem}
\begin{solution}
    In the $\xi = 0$ case, we immediately have that
    \[ (\SF f)(0) = \int_{-\infty}^\infty f(x) \, dx = \int_{-\infty}^\infty e^{-\pi x^2} = 1, \]
    from computation through the Gaussian integral (polar coordinate transform). If $\xi > 0$ (we note that $\xi < 0$ follows analogously), let $f : \CC \to \CC$ be given by $f(z) = e^{-\pi z^2}$. Then, for $R > 0$, let us integrate on the rectangle from $-R$ to $R$, $R$ to $R + i\xi$, $R + i\xi$ to $-R + i\xi$, and $-R + i\xi$ to $-R$, where the latter three paths are denoted $I_1, I_3, I_2$ respectively (we intentionally number this way because the two opposing sides $I_1, I_2$ can be tackled together at once). We have that
    \[ 0 = \int_{\gamma_R} f(z) \, dz = \int_{-R}^R f(x) \, dx + \int_{I_1} f(z) \, dz + \int_{I_2} f(z) \, dz + \int_{I_3} f(z) \, dz. \]
    Let us consider the integral $\abs{\int_{I_1} f(z) \, dz}$, with parameter $I_1 : [0, \xi] \to \CC$, given by $t \mapsto R + it$; we have that
    \[ \abs{\int_{I_1} f(z) \, dz} = \abs{\int_0^\xi f(R + it) i \, dt} = \abs{\int_0^\xi e^{-\pi(R + it)^2} i \, dt}, \]
    for which we observe that the integrand
    \[ \abs{e^{-\pi(R+it)^2} i} = \abs{e^{-\pi(R^2 - t^2)}} \abs{i} \abs{e^{i \pi 2 Rt}} \leq \abs{e^{-\pi(R^2 - \xi^2)}} \to 0, \quad R \to +\infty, \]
    so the integrals on $I_1, I_2 \to 0$ for large enough $R$ (we note that the same conclusion held for $I_2$ because the computation follows analogously). For the last part, consider that
    \[ \int_{I_3} f(z) \, dz = \int_{-R}^R f(t + i\xi) \, dt = \int_{-R}^R e^{-\pi(t + i\xi)^2} \, dt = e^{\pi \xi^2} \int_{-R}^R e^{-\pi t^2} e^{-2 \pi i \xi t} \, dt, \]
    upon which we obtain $e^{\pi \xi^2} (\SF f)(\xi)$ as $R \to \infty$. This means we have that $-e^{\pi \xi^2} (\SF f)(\xi)$ vanishes, where the minus sign is from the orientation of $I_3$, and so we may conclude that $(\SF f)(\xi) = e^{-\pi \xi^2} = f(\xi)$.
\end{solution}