\section{Day 9: Applications of Cauchy's Integral Formula (Sep. 30, 2025)}
Recall Liouville's theorem that if $f$ is an entire (holomorphic on the whole complex plane) bounded function, then $f$ is constant.
\begin{corollary}[Fundamental Theorem of Algebra]
    Every nonconstant polynomial $P(z) = a_nz^n + \dots + a_1z + a_0$ with complex coefficients has a root in $\CC$.
\end{corollary}
\begin{proof}
    Proceed by contradiction by means of Liouville's theorem. Suppose $P(z)$ is nonconstant and admits no roots in $\CC$; then $P(z)\inv$ is entire, and it remains to check that it is bounded. It is enough to get a lower bound for $a_nz^n$, since the dominating term of $P(z)$ is $a_nz^n$; supposing $a_n \neq 0$, we have that
    \[ \frac{P(z)}{z^n} = a_n + \left(\frac{a_{n-1}}{z} + \dots + \frac{a_n}{z^n}\right), \]
    of which we know is defined on $\CC \setminus \{0\}$; taking $\abs{z} \to +\infty$, we have that $\frac{P(z)}{z^n} \to a_n$, so there exists $R > 0$ such that
    \[ \abs{P(z)} \geq \frac{\abs{a_n}}{2} \abs{z}^n \]
    for all $\abs{z} > R$. This means
    \[ \frac{1}{\abs{P(z)}} \leq \frac{1}{\frac{\abs{a_n}}{2} \abs{z}^n} \leq \frac{1}{\frac{\abs{a_n}}{2} R^n}, \quad \text{for } \abs{z} > R. \]
    For any $z \in \ol{D_R(0)}$, we have that $P(z) \neq 0$. Since $P$ is continuous, there exists an open neighborhood $D_z$ of $z$ and $c_z > 0$ such that $\abs{P(z')} \geq c_z > 0$ for any $z' \in D_z$. Since $\ol{D_R(0)}$ is compact, there exists finitely many $D_{z_1}, \dots, D_{z_k}$ such that $\ol D_R(0) \subset \bigcup_{i=1}^k D_{z_i}$. Then $\abs{P(z)} \geq \min\{C_{z_1}, \dots, C_{z_k}\} > 0$ on $\ol{D_R(0)}$. Since we have a lower bound for $P(z)$ on the compact set $\ol{D_R(0)}$ and outside of it, we see that $P(z)\inv$ is bounded on $\CC$, and so per Liouville's theorem, $P(z)\inv$ is constant, yielding that $P(z)$ is constant, contradicting the assumption.
\end{proof}
\begin{corollary}
    Every polynomial $P(z) = a_nz^n + \dots + a_1z + a_0$ of degree $n \geq 1$ has precisely $n$ roots in $\CC$.
\end{corollary}
\begin{proof}
    Left as an exercise.
\end{proof}
\noindent We now discuss the applications of Cauchy's integral formula. Let $\{f_n\}_{n=1}^\infty$ be a sequence of holomorphic functions.
\begin{theorem}
    Let $\Omega$ be an open subset of $\CC$, and let $\{f_n\}_{n=1}^\infty$ be a sequence of holomorphic functions that converge uniformly to a function $f$ on every compact subest of $\Omega$. Then
    \begin{parlist}
        \item $f$ is holomorphic on $\Omega$,
        \item $\{f_n'\}_{n=1}^\infty$ converges uniformly to $f'$ on every compact subset of $\Omega$.
    \end{parlist}
\end{theorem}
\noindent We give some examples of such sequences.
\begin{enumerate}[(i)]
    \item Let $f_n(x) = \sqrt{x^2 + \frac{1}{n}}$ on $\RR$; we have that each $f_n$ is differentiable and $f_n(x) \to f(x) = \abs{x}$ as $n \to \infty$ on compact intervals, but $f(x)$ itself is not differentiable at $0$.
    \item (\textit{Weierstrass approximation theorem}) Every continuous function on a closed bounded interval $[a, b]$ can be uniformly approximate by a polynomial. Specifically, for every $\eps > 0$, there exists a polynomial $P(x)$ such that $\sup_{x \in [a, b]} \abs{f(x) - P(x)} < \eps$.
    \\[8pt]
    An additional side remark; $C([a, b])$, i.e., the set of all continuous functions on $[a, b]$, equipped with the uniform norm, has the set of polynomials dense in itself. 
\end{enumerate}
\begin{proof}
    We now prove the theorem. Let us start by showing that $\{f_n\}$ converges uniformly to $f$ on every compact subset of $\Omega$ and show that $f$ is holomorphic. The proof idea here is to use Morera's theorem to show that $f$ is continuous on a disc $D$, and so for every triangle $T \subset D$, we have that $\int_T f = 0$, meaning that $f$ is holomorphic. We may use Cauchy's theorem to see that each $\int_T f_n$ is equal to $0$, so $\int_T f = 0$, since
    \[ \abs{\int_T f_n - \int_T f} = \abs{\int_T f_n - f} \leq \sup_T \abs{f_n - f} \length(T) \taking{n \to \infty} 0. \]
    For the second part, we wish to show that $\{f_n'\}$ converges uniformly to $f'$ on every compact subset of $\Omega$. For $\delta > 0$, define $\Omega_\delta = \{z \in \Omega \mid \ol{D_\delta(z)} \subset \Omega\}$. Any compact subset of $\Omega$ is contained in some $\Omega_\delta$, so it suffces to show that $\{f_n'\}$ converges uniformly to $f'$ on $\Omega_\delta$ for each $\delta > 0$. We claim that if $F$ is holomorphic on $\Omega$, then
    \[ \sup_{z \in \Omega_\delta} \abs{F'} \leq \frac{1}{\delta} \sup_{z \in \Omega} \abs{F}. \]
    Applying the claim to $f_n - f$, we see that
    \[ \sup_{\Omega_\delta} \abs{f_n' - f'} \leq \frac{1}{\delta} \sup_\Omega \abs{f_n - f} \taking{n \to \infty} 0, \]
    so it remains to prove the claim itself. For all $z \in \Omega_\delta$, by Cauchy's integral formula for the derivative, we have that
    \[ F'(z) = \frac{1}{2\pi i} \int_{C_\delta(z)} \frac{F(\zeta)}{(\zeta - z)^2} d\zeta, \]
    where $C_\delta(z) = \{w \mid \abs{w-z}=\delta\}$, so for all $z \in \Omega_\delta$ we have,
    \[ \abs{F'(z)} \leq \frac{1}{2\pi} \sup_{\zeta \in C_\delta(z)} \abs{\frac{F(\zeta)}{(\zeta - z)^2}} \cdot 2\pi \delta = \frac{1}{\delta} \sup_{\zeta \in C_\delta(z)} \abs{F(\zeta)}, \]
    meaning we make take the supremum over $\Omega_\delta$ to get\footnote{page 54-55, shakarchi}
    \begin{align*}
        \sup_{z \in \Omega_\delta} \abs{F'(z)} &\leq \sup_{z \in \Omega_\delta} \left(\frac{1}{2\pi} \sup_{\zeta \in C_\delta(z)} \abs{\frac{F(\zeta)}{(\zeta - z)^2}} 2 \pi \delta\right) \\
        &\leq \sup_{z \in \Omega_\delta} \left(\frac{1}{\delta} \sup_{\zeta \in C_\delta(z)} \abs{F(\zeta)} \right) \\
        &\leq \frac{1}{\delta} \sup_{z \in \Omega} \abs{F(z)}. \qedhere
    \end{align*}
\end{proof}
\noindent The term test will be next Tuesday in class. It will be three problems; the first is to prove a theorem discussed in class, the second is a variation of a homework problem, and the third is a choice between another variation of a homework problem or a problem not in the homework, of which has higher marks (what?).
\newpage
\noindent We now discuss another application of Cauchy's integral formula; specifically, the Schwartz' reflection principle (Theorem 5.6 in Shakarchi, p. 60), which extends a holomorphic function analytically to a larger set. We start by presenting a counterexample.
\begin{theorem}[Fabry (Gap) Theorem]
    Consider a power series $f(z) = \sum_{k=0}^\infty a_{n_k} z^{n_k}$, where $\{n_k\}$ is a strictly increasing sequence of positive integers. Reference \href{https://en.wikipedia.org/wiki/Fabry_gap_theorem}{here}.
\end{theorem}
\noindent Suppose that $\frac{n_k}{k} \to \infty$ as $n \to \infty$, and the radius of convergence of the power series is $1$. Then $f$ cannot be analytically extended beyond any point of the unit circle. Let $z \in \partial \DD$; we want to show that we cannot find a holomorphic function $\tilde{f}$ defined on an open subset $U$ of $z$ such that
\[ \restr{\tilde f}{U \cap \DD} = \restr{f}{U \cap \DD}. \]
As an example, pick
\[ f(z) = \sum_{n=1}^\infty \frac{z^{n^2}}{n^2}, \]
for which $\frac{n^2}{n} = n \to \infty$ as $n \to \infty$. By Hadamard's formula, we have that the radius of convergence is indeed $1$. The computation showing that the analytic extension does not extend beyond the unit circle is left as an exercise.
\begin{theorem}[Schwartz reflection principle (Shakarchi 5.6)]
    Let $\Omega$ be an open subset of $\CC$ that is symmetric with respect to the real line, i.e., $z \in \Omega$ if and only if $\ol z \in \Omega$. Define $\Omega^+, \Omega^-$ to be subsets of $\Omega$ with positive and negative imaginary part respectively, and let $I = \Omega \cap \RR$. Let $f$ be a holomorphic function on $\Omega^+$ that extends continuously to $I$ and such that $f$ is real-valued on $I$. Then there exists $F$ holomorphic on all of $\Omega$ with $\restr{F}{\Omega^+ \sqcup I} = f$.
\end{theorem}
\noindent To do this, we start by defining a holomorphic function $F$ on $\Omega^-$, then we prove that $F$ is holomorphic on $\Omega$ (and $F$ is holomorphic on $I$).
\begin{theorem}[Symmetric principle]
    Let $f^+, f^-$ be holomorphic functions on $\Omega^+, \Omega^-$ respectively that extend continuously on $I$ such that they agree on $I$. Then the function $f$ on $\Omega$ defined by
    \[ f(z) = \begin{cases} f^+(z) &z \in \Omega^+, \\ f^+(z) = f^-(z) &z \in I, \\ f^-(z) &z \in \Omega^- \end{cases} \]
    is holomorphic on $\Omega$.
\end{theorem}
\noindent To see that $f$ is holomorphic on $I$, we may use Morera's theorem; pick any open disc $D$ centered at a point $z \in I$ which is entirely contained in $\Omega$. We will show that $f$ is holomorphic on $D$. Observe that any $T \subset D$ is of four types; either it \begin{parlist} \item does not intersect $I$, \item aligns with $I$ with one of its sides, \item intersects with $I$ at exactly one vertex, \item or intersects with $I$ at two points. \end{parlist} \\[8pt]
For case (i), we have that Cauchy's theorem immediately shows that $\int_T f(z) \, dz = 0$. For cases (ii) and (iii), we may let $T_\eps$ (i.e., moved upwards or downwards by $\eps$ so it is of case (i)) be an affine shift of $T$; then
\[ \int_{T_\eps} f(z) \, dz \taking{\eps \to 0} \int_T f(z) \, dz = 0. \]
For case (iv), we can partition the triangle into subtriangles satisfying case (ii) or (iii), and so we immediately have that the integral vanishes too.
\newpage
\noindent With this, we may now prove the Schwartz reflection principle.
\begin{proof}
    Let $f$ be holomorphic on $\Omega^+$ and let it extend continuously to $I$ such that it is real-valued on $I$. We claim that there exists $F$ on $\Omega$ such that the restrction of $F$ onto $\Omega^+$ is equal to $f$ on $\Omega^+$. We may construct such $F$ by having $f^-(z) = \ol{f(z)}$ for $z \in \Omega^-$. It suffices to check that $f^-$ is holomorphic on $\Omega^-$; for all $z, z_0 \in \Omega^-$, we have that $\ol z, \ol{z_0} \in \Omega^+$. Since $f$ is holomorphic at $\ol{z_0}$, $f$ admits a power series
    \[ f(\ol{z_0}) = \sum_{n=0}^\infty a_n(\ol z - \ol{z_0}), \]
    which converges on some $D_r(\ol{z_0})$ with $r > 0$. In particular,
    \[ f^-(z) = \ol{f(\ol z)} = \sum_{n=0}^\infty \ol{a_n} (z-z_0)^n, \]
    which converges on $D_r(z_0)$, which by Hadamard's formula, admits the same radius of convergence as the power series about $f(\ol{z_0})$, i.e., the power series for $f^-(z)$ converges on $D_r(z_0)$. Hence, $f^-$ is holomorphic at $z_0$, and since $f$ extends continuous to $I$ and is real valued on $I$, we have that $\ol{f(\ol x)} = f(x)$ for all $x \in I$, and so $f^-$ can be extended continuously to $I$ such that $f^- = f^+$ on $I$. In this manner, we may apply the symmetric principle from earlier to obtain $F$ satisfying the Schwartz reflection principle.\footnote{ref: p.60 shakarchi}
\end{proof}
