\section{Day 21: Harmonic functions, Pt.\ 2, and Harnack's Principle (Nov.\ 25, 2025)}
A continuous, real-valued function $u(z)$ on a region $\Omega \subset \CC$ is said to satisfy the mean value property
\[ u(z_0) = \frac{1}{2\pi} \int_0^{2\pi} u(z_0 + re^{i\theta}) \, d\theta \]
when the disc $\abs{z - z_0} \leq r$ is contained in $\Omega$. It was previously shown on problem set 2 that harmonic functions satisfy the mean value property. Recall, that the problem stated that harmonic functions satisfy the mean value property, and that harmonic functions also satisfy the max/min principle.
\begin{theorem}
    Any continuous real-valued function $u(z)$ on a region $\Omega$ which satisfies the mean value property is harmonic.
\end{theorem}
\noindent To prove this, we will use Poisson's formula and Schwarz's theorem to obtain solutions to Dirichlet's problem on discs. Recall,
\begin{theorem*}[Poisson's formula]
    Let $u$ be a harmonic function on $\abs{z} < R$ and continuous on $\abs{z} \leq R$. We have that
    \[ u(z) = \frac{1}{2\pi} \int_0^{2\pi} \frac{R^2 - \abs{z}^2}{\abs{Re^{i\theta} - z}^2} u(Re^{i\theta}) \, d\theta. \]
\end{theorem*}
\noindent Recall Dirichlet's problem; given a disc $\abs{z} \leq R$ and a function $u$ on $\abs{z} = R$, we want to find $U$ on $\abs{z} \leq R$ such that
\[ \begin{cases} \Delta u = 0, & \abs{z} < R, \\ U = u. & \text{on $\abs{z} = R$} \end{cases} \]
Given a (piecewise) continuous $u$ on $\abs{z} = R$, we have its Poisson integral is as follows,
\[ P_u(z) = \frac{1}{2\pi} \int_0^{2\pi} \frac{R^2 - \abs{z}^2}{\abs{Re^{i\theta} - z}^2} u(Re^{i\theta}) \, d\theta. \]
Indeed, $P_u$ is harmonic on $\abs{z} < R$. Observe that
\[ \frac{R^2 - \abs{z}^2}{\abs{Re^{i\theta} - z}^2} = \Re \left(\frac{Re^{i\theta} + z}{Re^{i\theta} - z}\right), \]
which we see from the computation
\[ \Re \left(\frac{Re^{i\theta} + z}{Re^{i\theta} - z}\right) = \Re \left(\frac{(Re^{i\theta} + z)(Re^{i\theta} - z)}{\abs{Re^{i\theta} - z}^2}\right) = \Re \left(\frac{(R^2 - \abs{z}^2 + Re^{i\theta} z - Re^{i\theta} z}{\abs{Re^{i\theta} - z}^2}\right). \]
Thus, our Poisson integral may be rewritten to be of the form
\[ P_u(z) = \Re\left(\frac{1}{2\pi} \int_0^{2\pi} \left(\frac{Re^{i\theta} + z}{Re^{i\theta} - z}\right) u(Re^{i\theta}) \, d\theta\right); \]
let the integrand be denoted $g(z, \theta)$ on $D_R(0) \times [0, 2\pi]$; clearly, $g$ is continuous on its domain, so $g(\cdot, \theta)$ is holomorphic on $D_R(0)$. This means that $P_u$ is the real part of a holomorphic function (since the integral is of a holomorphic function, it itself is also holomorphic), so $P_u$ is holomorphic.
\begin{theorem}[Schwarz's theorem]
    Given a continuous $u$ on $\abs{z} = R$, we have that $P_u$ is harmonic on $\abs{z} < R$. Moreover, for all $\zeta_0 \in \partial D_R(0)$, we have
    \[ \lim_{z \to \zeta_0} P_u(z) = u(\zeta_0). \]   
\end{theorem}
\begin{proof}
    Let $C_2$ be an open arc on the circle $\abs{z} = R$ about some fixed $\zeta_0$, and let $C_1$ be the complement of $C_2$ on said circle. Let us define $u_1 = u \circ \chi_{C_1}$ and $u_2 = u \circ \chi_{C_2}$. Indeed, we have $P_u(z) = P_{u_1}(z) + P_{u_2}(z)$; directly write as follows,
    \[ P_{u_1}(z) = \frac{1}{2\pi} \int_{\zeta \in C_1} \frac{R^2 - \abs{z}^2}{\abs{\zeta - z}^2} u(\zeta) \, d\theta \]
    as $z \to \zeta_0$; indeed, we have that $P_{u_1}(\zeta_0) = 0$ makes sense, since the denominator is nonvanishing on $C_1$, and that $P_{u_1}(z) \to P_{u_1}(\zeta_0)$ as $z \to \zeta_0 \in C_2$. Recall, that we want to show $\lim_{z \to \zeta_0} P_u(z) = u(\zeta_0)$; by replacing $u = u(\zeta_0)$, we assume that $u(\zeta_0) = 0$. Given any $\eps > 0$, there exists an open arc $C_2$ containing $\zeta_0$ such that $\abs{u(z)} < \eps/2$ for all $z \in C_2$; in particular, this means $\abs{u_1} < \eps/2$. Observe that, for any $\abs{z} < R$, we have that the integral of the Poisson kernel is given by
    \[ \frac{1}{2\pi} \int_0^{2\pi} \frac{R^2 - \abs{z}^2}{\abs{\zeta - z}^2} \, ds = 1(z) = 1, \]
    since the kernel itself is equal to the function sending everything to one.\footnote{?} Thus, we have that
    \[ \abs{P_{u_2}(z)} = \abs{\frac{1}{2\pi} \int_0^{2\pi} \frac{R^2 - \abs{z}^2}{\abs{\zeta - z}^2} u(\zeta) \, d\theta} < \frac{\eps}{2}. \]
    Thus, we conclude $P_u(z) < \eps$.
\end{proof}
\begin{theorem}
    A continuous real-valued function $u(z)$ in a region $\Omega \subset \CC$ which satisfies the mean value property is harmonic.
\end{theorem}
\noindent Pick any $z_0 \in \Omega$ and any disc $\abs{z - z_0} \leq r$ in $\Omega$. Then, by Schwarz's theorem,
\[ P_u(z) = \frac{1}{2\pi} \int_0^{2\pi} \frac{r^2 - \abs{z - z_0}^2}{\abs{re^{i\theta} - z}^2} u(z_0 + re^{i\theta}) \, d\theta \]
is harmonic on $\abs{z - z_0} < r$ and $P_u = u$ on $\abs{z - z_0} = r$. Thus, $u - P_u$ satisfies the mean value property by the max/min principle.
\begin{theorem}
    The Dirichlet problem can be solved for any open set $\Omega \subset \CC$ such that each boundary point is the endpoint of a line segment whose other points are exterior to $\Omega$.
\end{theorem}
\begin{proof}
    The proof is divided into three parts; Harnack's principle, subharmonic functions, and to solve a general version of the Dirichlet problem. Let $u$ be harmonic on $\abs{z} < R$ and continuous on $\abs{z} \leq R$; we have that, for all $\abs{z} \leq R$,
    \[ u(z) = \frac{1}{2\pi} \int_0^{2\pi} \frac{R^2 - \abs{z}^2}{\abs{Re^{i\theta} - z}^2} u(Re^{i\theta}) \, d\theta. \]
    Harnack's inequality states that, for $\rho := \abs{z} < R$,
    \[ \frac{R - \rho}{R + \rho} = \frac{R^2 - \rho^2}{(R + \rho)^2} \leq \frac{R^2 - \abs{z}^2}{\abs{Re^{i\theta} - z}^2} \leq \frac{R^2 - \rho^2}{(R - \rho)^2} = \frac{R + \rho}{R - \rho}. \]
    Thus,
    \[ u(z) \leq \frac{1}{2\pi} \cdot \frac{R + \rho}{R - \rho} \int_0^{2\pi} u(Re^{i\theta}) \, d\theta = \frac{1}{2\pi} \cdot \frac{R + \rho}{R - \rho} u(0). \]
    By the same way, for the other side of Harnack's inequality, we have that $u(z) \geq \frac{1}{2\pi} \cdot \frac{R - \rho}{R + \rho} u(0)$.
    \begin{theorem}[Harnack's principle]
        Consider a sequence of functions $u_n(z)$, each defined and harmonic in a certain region $\Omega_n$. Let $\Omega$ be a region such that every point in $\Omega$ has a neighborhood contained in all but finitely many $\Omega_n$. Assume, moreover, that in some neighborhood of $z \in \Omega$, we have that $u_n(z) \leq u_{n+1}(z)$ for all $n$ sufficiently large. Then there are only two possibilities for convergence: either $u_n(z) \to +\infty$, or $u_n(z)$ converges to a harmonic function on $\Omega_n$; in both cases, the converge is uniform on every compact subset of $\Omega$.
    \end{theorem}
    \begin{proof}
        Assume that, at one point $z_0 \in \Omega$, we have
        \[ \lim_{n \to \infty} u_n(z_0) = \infty. \]
        We will show that we get the first case. By assumption, there exists $r > 0$ such that $D_r(z_0) \subset \Omega_n$ for all $n \geq m$ (for some fixed $m$), and that the $u_n$ are harmonic, forming a non-decreasing sequence on $D_r(z_0)$. Applying Harnack's inequality, we have that, for all $z \in D_r(z_0)$,
        \[ \frac{1}{2\pi} \cdot \frac{R - \abs{z}}{R + \abs{z}} (u_n(z_0) - u_m(z_0)) \leq u_n(z) - u_m(z) \leq \frac{1}{2\pi} \cdot \frac{R + \abs{z}}{R - \abs{z}} (u_n(z_0) - u_m(z_0)).  \]
        Since the lower bound admits a term $u_n(z_0) - u_m(z_0)$ which goes to infinity, it is clear that $u_n(z)$ itself also goes to infinity by our inequality above. We've proven that for all $z \in D_r(z_0)$, $\lim_{n \to \infty} u_n(z) = \infty$. Let us consider the open set $\Omega_I = \{z \in \Omega \mid \lim_{n \to \infty} u_n(z) = \infty\}$. Indeed, $\Omega_I$ contains $D_r(z_0)$; we will show that $\Omega \setminus \Omega_I = \{z \in \Omega \mid \lim_{n \to \infty} u_n(z) < \infty\}$ is also open. By considering the upper bound from Harnack's inequality, we see that $u_n(z_0)$ is a Cauchy sequence, which means $u_n(z) - u_m(z)$ is also Cauchy, so we are done.
    \end{proof}
    \noindent We will finish the rest of the proofs next lecture.\footnote{these are from ahlfors; i will annotate more once we write the compilation}
\end{proof}