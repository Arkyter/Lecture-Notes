\section{Day 30: Toward's Green's Function (Feb. 4, 2026)}

\begin{lemma}[Representation Formula]
		\( u \in \mathcal{C}^{2} \) and \( x_{0} \in D \), \( \Delta u = 0 \) then
	\[ \mu(x_{0}) = \int_{D} v(x) (- \Delta) u(x) dx + \int_{\partial D}  \mu \frac{\partial}{\partial n} v - v \frac{\partial u}{\partial n} dS_{x} \]
	where
	\[ v_{x_{0}}(x) - \frac{1}{4 \pi} \frac{1}{|x - x_{0}|} \]
\end{lemma}
\begin{proof}
	We did it with \( \Delta u = 0 \). As an exercise we want to add \( \Delta u \ne 0 \) to that proof.
\end{proof}

We used \[ v_{x_{0}}(x) = -\frac{1}{4 \pi |x - x_{0}|} \]
as a very important function. This is because intuitively it has the property that
\[ \int_{D} - \Delta v u dx = u(x_{0}), \]
even though this is not well defined, so in some sense \( v(x, x_{0}) \) solves \( \Delta_{x} v_{x_{0}}(x) = \delta_{x_{0}}(x) \). 

\begin{remark}
	We saw a similar idea with the heat kernel where it approximates the delta function. Also we saw it with the Poisson kernel.
\end{remark}

We'll use our representation formula to solve
\[ \left\{\begin{NiceMatrix}- \Delta u = 0 & \text{in \( D \)} \\ u = h & \text{on \( \partial D \)}.\end{NiceMatrix}\right.  \]
for a general domain \( D \).

\begin{definition}
	A Green's function for \( - \Delta \) in a domain \( D \) at a point \( x_{0} \in D \) is a function defined such that \( \forall x \in D \) such that
	\begin{enumerate}
	
		\item \( G \in \mathbb{C}^{2} \) with \( \Delta_{x} G = 0 \) in \( D \)
		\item \( G(x) = 0 \) on \( \partial D \)
		\item \( G(x) + \frac{1}{4 \pi |x - x_{0}|}\) is finite at \( x_{0} \) and \( \mathcal{C}^{2} \) everywhere, and harmonic everywhere.
	
	\end{enumerate}

\end{definition}

	\begin{remark}
		Varying \( x_{0} \) we get a two parameter function \( G(x, x_{0}) \) \( x \ne x_{0} \). This will be notation used henceforth.
	\end{remark}

	\begin{theorem}
		If \( G(x, x_{0}) \) is a Green's function for a domain \( D \), then
		\[ u(x_{0}) = \int_{\partial D} \frac{\partial G}{\partial n}(x, x_{0}) u(x) dS_{x}.  \]
		solves the problem
		\[ \left\{\begin{NiceMatrix} \Delta u = 0 & \text{in \( D \)} \\ u = h & \text{on \( \partial D \)}\end{NiceMatrix}\right.  \]

		If we have the alternative condition  that
		\[ - \Delta u = f, \]
		this adds
		\[ \int_{D} f(x) G(x, x_{0}) dx \]
		to our solution.
	\end{theorem}
	\begin{proof}
		Recall the representation formula above.
		Define \( H \) by \( G(x, x_{0}) = v_{x_{0}}(x) + H(x) \). Then by our properties of Greens' functions, we have that \( \Delta H = 0 \) everywhere (specifically, \( (i) \) gives it everywhere away from \( x_{0} \) and \( (iii) \) tells us at \( x_{0} \)). From Green's second identity on \( u \) and \( H \), we have that
		\[ 0 = \int_{\partial D} u \frac{\partial H}{\partial n} - H \frac{\partial u}{\partial n} dS_{x}. \]
		Therefore, if we add this to our representation formula we get
		\begin{align*}
			u(x_{0}) &= \int_{\partial D}  u \frac{\partial v}{\partial n} - v \frac{\partial u}{\partial n} dS_{x} + \int_{\partial D}  u \frac{\partial H}{\partial n} - H \frac{\partial u}{\partial n} dS_{x} \\
							 &= \int_{\partial D} u(x) \frac{\partial G_{x_{0}}(x)}{\partial n}(x) - G_{x_{0}}(x) \frac{\partial u}{\partial n}(x) dS_{x}
		\end{align*}
		and the second term cancels by property \( 2 \).
	\end{proof}

	In a bit we'll find Green's function for some domain (half-space, ball). Before this let's show that
	\begin{lemma}
		For Green's function \( G(x, x_{0}) \), \( x, x_{0} \in D \), \( x \ne x_{0} \), then
		\[ G(x, x_{0}) = G(x_{0}, x). \]
	\end{lemma}
	\begin{proof}
		Consider \( u(x) = G(x, a) \) and \( v(x) = G(x, b) \) for \( a, b \in D \). Then Green's second identity tells us that
		\begin{align*}
			0 &= \int_{D \setminus (B_{\varepsilon}(a) \cup B_{\varphi}(b)}  u \Delta v - v \Delta u dx \\
				&= \int_{\partial \left( B_{\varepsilon}(a) \cup B_{\varepsilon}(b) \right)} u \frac{\partial v}{\partial n} - v \frac{\partial u}{\partial n} dS
				\intertext{because \( G|_{x \in \partial D} = 0 \)}
				&= \int_{\partial B_{\varepsilon}(a)} u \frac{\partial v}{\partial n} - v \frac{\partial u}{\partial n} dS + \int_{\partial B_{\varepsilon}(b)} u \frac{\partial v}{\partial n} - v \frac{\partial u}{\partial n} dS  
		\end{align*}
		Recall that \( u \) is singular at \( x = a \) and \( v \) is singular at \( x = b \). Then,
			\[ 0 = \lim_{\varepsilon \to 0} \int_{\partial B_{\varepsilon}(a)} - v \frac{\partial u}{\partial n} dS + \int_{\partial B_{\varepsilon}(b)} u \frac{\partial v}{\partial n} dS. \]  
			Now, we ask where do these different pieces converge to as \( \varepsilon \to 0 \)? Well
			\[ \lim_{\varepsilon \to 0} \int_{\partial B_{\varepsilon}(a)} - v \frac{\partial u}{\partial n} dS \xrightarrow{\varepsilon \to 0} v(a).  \]
			This is because we're basically just taking the average on a sphere of the values of our \( v \). This is the same proof as last class.
	\end{proof}

Let's solve Dirichlet's problem in half-space. We'll use Green's function method. 
\[ \left\{\begin{NiceMatrix} \Delta u = 0 & \text{in \( z > 0 \)} \\ u = h(x, y) & \text{on \( z = 0 \)}\end{NiceMatrix}\right.  \]
\( (x, y) \in \mathbb{R}^{2} \) coord on the boundary \( P = (x_{0}, y_{0}, z_{0}) \) a point in \( z > 0 \). We're looking for a Green's function. We want \( G(x, x_{0}) \) defined at all \( x \ne x_{0} \) in \( z > 0 \).
\begin{enumerate}

	\item \( G \in C^{2} \) when \( x \ne x_{0} \)
	\item \( G(x, x_{0}) = 0 \) when \( x \in \partial D \)
	\item \( G(x, x_{0}) + \frac{1}{4 \pi |x-x_{0}|} \) is \( C^{2} \) and harmonic.

\end{enumerate}
If we let \( x_{0}^{*} \) be the reflection over the \( z \) axis of \( x_{0} \), then we can write
\[ - \frac{1}{4 \pi |x - x_{0}|} + \frac{1}{4 \pi |x - x_{0}^{*}|}. \]
Such a reflection is gonna be anti-symmetric over the \( z \)-axis, so we will have \( 0 \) on the boundary. Also we've already shown it satisfies property one. For property three, notice that this is
\[ G(x, x_{0}) + \frac{1}{4 \pi |x - x_{0}|} = \frac{1}{4 \pi |x - x_{0}^{*}|} \]
which we know is \( C^{2} \) and harmonic away from \( x_{0}^{*} \), which is outside of our domain \( D \).

Therefore, the solution to the problem is
\[ u(x_{0}) = \int_{z = 0} \frac{\partial G}{\partial n}(x, x_{0}) \cdot h(x) dS_{x} \]
writing this out explicitly, we have that
\[ \left.\frac{\partial G}{\partial n}\right|_{z = 0} = \underbrace{2}_{\text{\( G \) is odd}} \cdot \left( \frac{1}{2} \frac{2 (z - z_{0})}{4 \pi |x - x_{0}|^{3}} \right) = \frac{-z_{0}}{2 \pi |x-x_{0}|^{3}}. \]

	Plugging back in, when \( x_{0} \in \left\{ z > 0 \right\} \),
	\[ u(x_{0}) = \frac{- z_{0}}{2 \pi} \int_{z = 0}  \frac{1}{|x_{0} - x'|^{3}} h(x') dx'. \]
	Next time we'll do the ball in \( \mathbb{R}^{3} \). The exercise is to apply Green's functions method to the half-plane.

	\begin{remark}
		So far, all the procedures and ideas are general, except the only specific thing we used is
		\[ v_{x_{0}}(x) = - \frac{1}{4 \pi |x - x_{0}|} \]
		as the starting point. This was because we are in 3d. In 2d we'd do the same thing but with a different \( v_{x_{0}} \). Specifically, in 2d, our fundamental solution is going to look something like \( \log |x - x_{0}| \). This is because in 3d the radial solution was the above \( v \). This might need a coefficient and such. Next time try applying it to the disk to get the Poisson kernel.
	\end{remark}
