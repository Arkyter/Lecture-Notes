\section{Day 12: Heat/Diffusion Part 3 (Oct. 15, 2025)}

\begin{theorem}
	Let \( \phi \) be bounded and continuous, \( \phi = \phi(x) \), \( x \in \mathbb{R} \). Then the formula \( u(x, t) = \int_{\mathbb{R}} S(x-y, t) \phi(y) dy \), where
	\[ S(z, t) = \frac{1}{\sqrt{4 \pi t}}e^{- \frac{z^{2}}{4t}}, \]
	is such that
	\begin{enumerate}
	
		\item \( u \in C^{\infty}(\mathbb{R} \times (0, \infty)), \)
		\item \( u \) solves \( u_{t} - u_{x x} = 0 \).
		\item \( \lim_{t \searrow 0} u(x, t) = \phi(x) \) for all \( x \).
	
	\end{enumerate}
	Notably, even if \( \phi \) is not smooth, \( u \) is.
\end{theorem}
\begin{proof}
	We can write 
	\[ u(x, t) = \int_{\mathbb{R}} S(z, t) \phi(x - z) dz. \]
	This is just the convolution \( S( \cdot, t) * \phi \). Now, if we set \( z = \sqrt{t p} \), then  this yields
	\[ u(x, t) \int_{\mathbb{R}} e^{- \frac{p^{2}}{x}} \phi(x - \sqrt{t}p) dp. \]
	Notice that \( |u(x, t)| \le \sup_{y}| \phi(y)| \).
	\begin{enumerate}
	
		\item To prove that \( u \in C^{\infty}(\mathbb{R} \times (0, \infty)) \), first we will state a nice general lemma.
			\begin{lemma}
				\[ \lim_{m \to \infty} \int_{\mathbb{R}} F_{m}(y) dy = \int_{\mathbb{R}} \lim_{m \to \infty} F_{m}(y) dy, \]
				provided that \( \lim F_{m}(y) \) exists for all \( y \in \mathbb{R} \), this limit is absolutely integrable, and \( F_{m} \rightrightarrows F \).
			\end{lemma}
			From here, notice that if \( F_{m}(x) = \frac{u(x + h_{m}, t) - u(x, t)}{h_{m}} \), then applying this to suitable sequences \( h_{m} \) yields that
		\[ \frac{\partial}{\partial x}u(x, t) = \int_{\mathbb{R}} \frac{\partial}{\partial x} S(x - y, t) \phi(y) \, dy \]
			To do this, let us check our conditions. For absolute integrability,
			\begin{align*}
				\int_{\mathbb{R}} \frac{\partial}{\partial x} S(x - y, t) \phi(y) \, dy	&= \frac{1}{\sqrt{4 \pi t}} \int_{\mathbb{R}} - \frac{x - y}{2t} e^{- \frac{(x - y)^{2}}{4t}} \phi(y) dy \\
				\tag{\( \frac{x - y}{\sqrt{t}} = p \)}&= \frac{1}{\sqrt{4 \pi t}} \int_{\mathbb{R}} \frac{1}{2 \sqrt{t}} e^{- p^{2}}p \phi(x - p \sqrt{t}) dp. \\
																							&\implies \left| \int_{\mathbb{R}} \frac{\partial}{\partial x} S(x-y, t) \phi(y) dy \right| \le \frac{C}{t} \sup | \phi|.
			\end{align*}
			This proves that this middle part is absolutely integrable. Also, by Taylor's formula,
			\begin{align*}
				\tag{For some \( 0 \le |\xi| \le |h_{m}| \)}\left| \frac{S(z + h_{m}, t) - S(z, t)}{h_{m}} - \frac{\partial}{\partial z}S(z, t) \right|  &= \left| \frac{\partial}{\partial z}S(z + \xi, t) - \frac{\partial}{\partial z} S(z, t) \right| \\
				\tag{\( 0 \le |\xi_{0}| \le | \xi| \)}& = | \xi| \cdot |S''(z + \xi_{0}, t)| \\
				&\le |h_{m}| \sup |S''(z, t)| \\
			\end{align*}
			and thus we have that our convergence is uniform. It follows that we can interchange the integral and limit. It follows that the first derivative of \( u \) exists. We can do similar proofs to show it is \( C^{k} \) for any \( k \), so it is \( C^{\infty} \).
		\item In the proof of \( (i) \) we showed that
			\[ \frac{\partial^{2}}{\partial x^{2}} u(x, t) = \int_{\mathbb{R}} \frac{\partial^{2}}{\partial x^{2}} S(x - y, t) \phi(y) dy. \]
			We also showed that
			\[ \frac{\partial}{\partial t} u(x, t) = \int_{\mathbb{R}}  \frac{\partial}{\partial t}S(x - y, t) \phi(y) dy \]
			thus, since \( S \) solves the heat equation for \( t > 0 \),
			\[ u_{t} - u_{x x} = 0. \]

			\item We want to show that \( \lim_{t \searrow 0} u(x, t) = \phi(x) \).
				\[ u(x, t) - \phi(x) = \int_{\mathbb{R}} \frac{1}{\sqrt{4 \pi}}e^{- \frac{p^{2}}{4}} \phi(x - \sqrt{t}p) - \frac{1}{\sqrt{4 \pi}} e^{-\frac{p^{2}}{2}} \phi(x) \, dp, \]
				so we need to show that
				\[ \lim_{t \searrow 0} \int_{\mathbb{R}} e^{- \frac{p^{2}}{4}} \left[ \phi(x - \sqrt{t}p) - \phi(x) \right] dp = 0. \]
				To do this, we can just use the continuity of \( \phi \). It \( \sqrt{t}p \ll 1 \), this should be roughly \( 0 \) close to \( x \), and \( e^{- \frac{p^{2}}{4}} \) makes things small elsewhere. Specifically, 
				\[ \int_{\mathbb{R}} e^{- \frac{p^{2}}{4}} \left[ \phi(x - \sqrt{t}p) - \phi(x) \right] dp = I_{1} + I_{2}, \]
				where we have that
				\begin{align*}
					I_{1} &= \int_{|p \sqrt{t}| < \delta} e^{- \frac{p^{2}}{4}} \left[ \phi(x - \sqrt{t}p) - \phi(x) \right] dp \\
					I_{2} &= \int_{|p \sqrt{t}| \ge \delta} e^{- \frac{p^{2}}{4}}\left[ \phi(x - \sqrt{t}p) - \phi(x) \right] dp.
				\end{align*}
				Fix \( \varepsilon > 0 \) arbitrary. We want to show that there exists \( t_{0} > 0 \) such that for all \( t \le t_{0} \) we have that
				\[ |I_{1}| < \frac{\varepsilon}{2}, \quad |I_{2}| < \frac{\varepsilon}{2}. \]
				To do this, let \( \delta \) be such that
				\[ | \phi(x - y) - \phi(x)| < \frac{\varepsilon}{2}, \qquad |y| < \delta. \]
				\begin{enumerate}
				
					\item To estimate \( I_{1} \), notice that
						\begin{align*}
							|I_{1}| &\le \int_{|\sqrt{t} p| < \delta} e^{- \frac{p^{2}}{4}} | \phi(x - \sqrt{t}p) - \phi(x)| \, dp \\
							&\le \int_{\mathbb{R}} e^{- \frac{p^{2}}{4}} \frac{\varepsilon}{2 \sqrt{4 \pi}} \, dp \le \frac{\varepsilon}{2}.
						\end{align*}
					\item To estimate \( I_{2} \),
						\begin{align*}
							|I_{2}| &\le \left( 2 \sup_{y \in \mathbb{R}} \phi(y) \right) \int_{|p \sqrt{t}| \ge \delta} e^{- \frac{p^{2}}{4}} \, dp \\
											&=\left( 2 \sup_{y \in \mathbb{R}} \phi(y) \right) \int_{|p| < \frac{\delta}{\sqrt{t}}} e^{- \frac{p^{2}}{4}}\, dp  \\
											&= 4 \sup | \phi| \int_{\frac{\delta}{\sqrt{t}}}^{\infty} e^{- \frac{p^{2}}{4}}\, dp. 
						\end{align*}
						Given \( \varepsilon \) and \( \delta \), pick \( t_{0} \) small enough such that
						\[ \int_{\frac{\delta}{\sqrt{t_{0}}}}^{\infty} e^{- \frac{p^{2}}{4}}\, dp < \frac{\varepsilon}{8 \sup | \phi|}, \]
						so we are done.
				\end{enumerate}
	\end{enumerate}
\end{proof}

\begin{remark}
	We always said that \( u(x, t) = S(\cdot, t) * \phi \). This is \textbf{a} solution to the IVP
	\[ \left\{\begin{NiceMatrix}u_{t} = u_{x x} & t > 0 & x \in \mathbb{R}\\ u = \phi & t = 0 & x \in \mathbb{R}.\end{NiceMatrix}\right.  \]
	We said this because it's not unique. In particular, there exists \( u \) solving
	\[ \left\{\begin{NiceMatrix}u_{t} = u_{x x} & t > 0 \\ u(x, 0) = 0,\end{NiceMatrix}\right.  \]
	but with \( u \not \equiv 0 \). At the same time, under some conditions (see homework) we can get uniqueness.
\end{remark}

\begin{remark}
	\( u_{t} = u_{x x} \) is ill-posed for \( t < 0 \). To see this, consider the data
	\[ \frac{1}{m} \sin(m x) = \phi_{m}(x). \]
	Then the solution is
	\[ u_{m}(x, t) = \frac{1}{m} \sin(mx) e^{- m^{2} t}. \]
	If \( t = - \varepsilon \), then \( u_{m}(x, - \varepsilon) = \frac{1}{m}\sin(mx) e^{\varepsilon m^{2}}, \)
	so there's no continuous dependence on data.
\end{remark}

\begin{example}[Diffusion with a source]
	This is the non-homogeneous case.
	\[ u_{t} - u_{x x} = f(x, t). \]
	We'll say that \( t > 0 \), and that \( f \) has to be nice enough such that the following stuff still works. We want to find a solution using Duhamel's formula/operator method. To do this, we want to ``think of the pde as an ode.'' To do this, recall how Duhamel works.
	\[ \left\{\begin{NiceMatrix}u_{t} + Au = f \\ u(0) = \phi,\end{NiceMatrix}\right.  \]
	where \( A \) is a matrix (linear operator). Use integrating factors.
	\[ v = e^{At}u \implies v_{t} = e^{At}f. \]
	This yields that
	\begin{align*}
		v(t) &= \underbrace{v(0)}_{\phi} + \int_{0}^{t} e^{As}f(s) \, ds \\
		u(t) &= e^{-At} \phi + \int_{0}^{t} e^{-A (t - s)} f(s) \, ds.
	\end{align*}
	In the ODE case, \( e^{-A t} \phi \) is the solution of the homogeneous problem. In the PDE case, this is given by
	\[ e^{-At \phi} \leftrightarrow S_{t} * \phi. \]
	We can also write this as \( e^{t \partial_{x}^{2}} \phi \) [I do hate this, sorry Fabio]. Thus, writing our Duhamel's formuls for PDEs yields
	\[ u(x, t) = S_{t} * \phi + \int_{0}^{t} S_{t-s}( \cdot) * f( \cdot, s) ds. \]
	We will work with this more next time.

\end{example}

