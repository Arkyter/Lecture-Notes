\section{Day 18: Fourier Series Part 2. (Nov. 12, 2025)}
Solving PDEs on bounded domains (intervals for now) with boundary conditions led to Fourier Series.
\begin{example}
	We would like to represent a function \( f(x) = \sum_{n \in \mathbb{Z}} c_{n} X_{n}(x) \) where \( X_{m} \) solves the Eigenvalue Problem
	\[ \left\{\begin{NiceMatrix}- X''_{n} = \lambda_{n} X_{n} \\ \text{Boundary Conditions}\end{NiceMatrix}\right.. \]
\end{example}

\begin{remark}
	If we want to write \( f(x) = \sum_{n=0}^{\infty} c_{n} \sin \left( \frac{n \pi}{\ell} x \right), \), then we can do this by integrating against \( \sin \left( \frac{m \pi}{\ell} x \right) \) so long as we can switch the integral with our sum. This will yield
	\[ c_{n} = \frac{2}{\ell} \int_{0}^{\ell}  f(x) \sin \left( \frac{\pi n}{\ell} x \right) dx. \]
	Furthermore, if this works, then it's the unique coefficient. This used the orthogonality of our sine functions.
\end{remark}

If we are given \( f \) which is a function on \( [0, \ell] \) and define 
\[ c_{n} = \int_{0}^{\ell}  f(x) X_{n}(x) dx, \]
where \( X_{n} \) satisfies our eigenvalue problem. Is it the case that
\[ f(x) = \sum_{n}  c_{n} X_{n}(x)? \]
Well there's a limit here in our sum, so we have to answer with respect to which function space are we considering the convergence? From here we can ask which class of functions we can pick \( f \) from to make it work. Notice that it's also possible that our sum converges, but not to \( f \). Let's start with a version that is a bit abstract.

\subsection{Orthogonality and Symmetric Boundary Conditions}
Fiven two functions \( f, g \) on \( [a, b] \), define an inner product for functions
\[ \left( f, g \right) = \int_{a}^{b} f(x) g(x) dx. \]
[He did indeed use the round parentheses here. Come on Fabio.] If we are using Complex Valued functions, we should use the conjugate on the second function to make it a Hermitian inner product. Refer back to 247. We say \( f \perp g \) if \( \left( f, g \right) = 0 \). Refer back to 247.

\begin{question}
	What kind of functions are \( f \) and \( g \)? If \( f \) and \( g \) are continuous, then this will be an inner product, but if we just require integrable, then we might have that there exists \( f \ne 0 \) but \( (f, f) = 0  \), which would contradict our inner product axioms. 

	An example of such a function is the indicator function on a singleton. But for the sake of our theory, instead of worrying about this, we will think of functions as being parts of equivalence classes, where
	\[ f \sim g \iff \left( f - g, f - g \right) = 0. \]
	This turns out to be equivalent to \( ||f - g||_{L^{2}} = 0 \), which is fun. [Fabio defined it in class as \( f - g \) differing only on a set with measure zero, but I like this definition more because it is more direct in my humble opinion].
\end{question}

\begin{definition}
	We say that a Hilbert space \( H \)	 is a vector space with an inner product \( \left\langle \cdot, \cdot \right\rangle \)such that \( H \) is complete with respect to the norm \( ||f|| = \left\langle f, f \right\rangle^{\frac{1}{2}} \).\footnote{Isaac wants to point out that ``erm technically[\ldots]'' I will not repeat any such remarks here. You, dear reader, had the choice to take 436 if you wanted to.} Completeness here is defined such that Cauchy sequences with respect to \( || \cdot || \) converge in \( H \).
\end{definition}
Hilbert spaces have lots of geometric properties. They are very similar to finite dimensional space. They are ``closest to finite dimensional''. The problem is that in infinite dimensions we lose compactness of the unit closed ball. In this general context, define \( v \perp w \) if \( \left\langle v, w \right\rangle = 0 \). Let us go over properties of Hilbert spaces.

\begin{lemma}
	\begin{enumerate}
	
		\item Cauchy-Schwarz holds for Hilbert Spaces.
			\[ \left| \left\langle u, v \right\rangle \right| \le ||u|| \, ||v|| \]
		\item The Parallelogram Identity holds for Hilbert Spaces.
			\[ ||u+v||^{2} + ||u-v||^{2} = 2 \left( ||u||^{2} + ||v||^{2} \right). \]
		\item Pythagorean Theorem holds for Hilbert Spaces.
			\[ u \perp v \implies ||u + v||^{2} = ||u||^{2} + ||v||^{2}. \]
	
	\end{enumerate}
\end{lemma}

\begin{proposition}
	If \( \left( w_{n} \right)_{n \ge 1} \) is an orthonormal sequence---namely, \( \left\langle w_{n}, w_{m} \right\rangle = \delta_{n, m}, \)	then for all \( v \),
	\[ \sum \left\langle v, w_{n} \right\rangle^{2} \le ||v||^{2}. \]
\end{proposition}

\begin{corollary}
	If we look at the series \( \sum \left\langle v, w_{n} \right\rangle w_{n} \), then this converges in \( H \).
\end{corollary}
\begin{proof}
	This is because our sum is absolutely convergent.
\end{proof}
