\section{Day 57: Integration of Differential Forms and Stokes' Theorem (Feb. 28, 2025)}
Let $\omega$ be a $\SC^0$ $1$-form defined on an open subset $U \subset \RR^n$. Let $\gamma : [0, 1] \to U$ be a $\SC^1$ curve, and
\[ \int_\gamma \omega = \int_{[0, 1]} \gamma^\ast \omega. \]
If $\omega = df$ for some $\SC^1$ $f$, then $\int_\gamma \omega = \int_{\partial \gamma} f$.
\begin{definition}
    Let $\gamma : [0, 1] \to U$ is said to be piecewise $\SC^1$ if $\gamma$ is continuous, and there is a partition $P = \{t_0, \dots, t_n\}$ of $[0, 1]$ such that $\restr{\gamma}{[t_{i-1}, t_i]}$ is $\SC^1$ for each $i$.
\end{definition}
\noindent In this situation, if $\omega = df$, then
\[ \int_\gamma \omega = \sum_{i=1}^n \int_{\restr{\gamma}{[t_{i-1}, t_i]}} \omega = \sum_{i=1}^n f\left(\gamma(t_i) - \gamma(t_{i-1})\right) = \int_{\partial \gamma} f. \]
We now generalize this to higher dimensions. Let $A \subset \RR^n$, and let $c : [0, 1]^k \to A$ be a singular $k$-cube in $A$. Note that by convention, $[0, 1]^0$, i.e. a $0$-cube in $A$, is just a point in $A$. We denote the \textit{standard} $k$-cube to be
\begin{align*}
    I^k : [0, 1]^k &\to \RR^k \\
    x \mapsto x.
\end{align*}
\vspace{-0.37in}
\begin{definition}
    A $k$-chain $c$ is a formal sum of $k$-cubes, i.e. an element of the free module over $\ZZ$ generated by singular $k$-cubes. Note that $1 \cdot c = c$.
\end{definition}
\noindent Given a $k$-chain $c$, we have that the boundary is, intuitively, a $(k-1)$-chain $\partial c$. To see this, let us build the definition from that of a standard $k$-cube. We have that the faces of $I^k$ can be regarded as $(k-1)$-cubes $I^k_{(i, \alpha)}$, where $i = 1, \dots, k$, and $\alpha = 0, 1$, defined by
\[ I^k_{(i, \alpha)} (x_1, \dots, x_{k-1}) = I^k (x_1, \dots, \alpha, \dots, x_{k-1}), \]
where $\alpha$ resides in the $i$th index. With this, we may define
\[ \partial I^k = \sum_{i=1}^k \sum_{\alpha = 0, 1} (-1)^{i + \alpha} I^k_{(i, \alpha)}. \]
Now, for a singular $k$-cube $c : [0, 1]^k \to A$, the $(i, \alpha)$ face is given by the composition $c_{(i, \alpha)} = c \circ I_{(i, \alpha)}$, and so the boundary is simply given as,
\[ \partial c = \sum_{i=1}^n \sum_{\alpha = 0, 1} (-1)^{i+\alpha} c_{(i, \alpha)}; \]
finally, for a $k$-chain $c = \sum_i a_ic_i$, we have that the boundary is given by $\partial c = \sum_i a_i \partial c_i$.
\begin{simpleprop}[Spivak 4-12]
    If $c$ is a $k$-chain in $A$, then $\partial(\partial c) = 0$, i.e. $\partial^2 = 0$.
\end{simpleprop}
\noindent The proof is left as an exercise in computation (it is also proven in the book). However, note that its converse is not necessarily true; given $c$ such that $\partial c = 0$, it is not necessary for there to exist such that $c = \partial d$; the answer depeneds on $A$, and generally no such $d$ exists. For example, let $c : [0, 1] \to \RR^2 \setminus \{0\}$ be given by $c(t) = (\sin 2\pi n t, \cos 2\pi n t)$, with $n > 0$. Then $c(1) = c(0)$, and so $\partial c = 0$, but no such $2$-chain $c'$ exists in $\RR^2 \setminus \{0\}$ with $\partial c' = c$; in general, $A$ being simply connected is sufficient, as closed implies exact on such $A$.
\medskip\newline
We now discuss the integral of a continuous $k$-form $\omega$ in $\RR^n$ over a standard $k$-cube $I^k$. Let $\omega = f(x) \, dx_1 \wedge \dots \wedge \dots dx_k$. We define,
\[ \int_{I^k} \omega = \int_{[0, 1]^k} \omega = \int_{[0, 1]^k} f, \]
i.e.,
\[ \int_{[0, 1]^k} f(x) \, dx_1 \wedge \dots \wedge dx_k = \int_{[0, 1]^k} f(x_1, \dots, x_k) \, dx_1 \wedge \dots \wedge dx_k. \]
If $\omega$ is a continuous $k$-form on $A \subset \RR^n$, $c$ is a singular $k$-cube in $A$, then we define
\[ \int_c \omega = \int_{[0, 1]^k} c^\ast \omega. \]
We discuss more next class.