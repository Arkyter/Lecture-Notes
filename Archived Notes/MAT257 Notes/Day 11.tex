\section{Day 11: Computations of Derivatives (Sep. 27, 2024)}
We start with a few useful facts: for the following examples, we let $f : (U \subset \RR^n) \to \RR^m$.
\begin{enumerate}[label=(\alph*)]
    \item If $f$ is a constant function $f = b$, then $D f(a) = 0$.
    \[ \lim_{h \to 0} \frac{f(a + h) - f(a) - 0}{\abs{h}} = \lim_{h \to 0} \frac{b - b}{\abs{h}} = 0. \]
    \item If $f$ is a linear transformation, then $D f(a) = f$. 
    \[ \lim_{h \to 0} \frac{f(a + h) - f(a) - f(h)}{\abs{h}} = \lim_{h \to 0} \frac{f(a + h) - f(a + h)}{\abs{h}} = 0. \]
    \item If $f = (f_1, \dots, f_m)$, $f$ is differentiable at $a$ if and only if each $f_i$ is differentiable at $a$, and $D f(a) = (D f_1(a), \dots, D f_m(a))$. We call $D f(a)$ to be the Jacobian (read: $m \times n$ matrix) where the $i$th row is given by $D f_i(a)$. \footnote{check this... i'm not so sure..? nvm \href{https://en.wikipedia.org/wiki/Jacobian_matrix_and_determinant}{jacobian wikipedia link}}
    \begin{itemize}
        \item[$(\Rightarrow)$] Observe that if $f$ is differentiable at $a$, then $f_i = \pi_i \circ f$, where $\pi_i$ is the projection from $\RR^m \to \RR$, which is differentiable by the chain rule.
        \item[$(\leftarrow)$] Now suppose each $f_i$ is differentiable at $a$; write $\lambda = (Df_1(a), \dots, Df_m(a))$, and consider
        \begin{align*}
            & \lim_{h \to 0} \frac{f(a + h) - f(a) - \lambda(h)}{\abs{h}} \tag{Expand component-wise} \\
            &= \lim_{h \to 0} \frac{\left(f_1(a + h) - f_1(a) - Df_1(a), \dots, f_m(a + h) - f_m(a) - Df_m(a)\right)}{\abs{h}} = 0
        \end{align*}
        \[  \]
        by the triangle inequality. \qed
    \end{itemize}
    \item Suppose $S : \RR^2 \to \RR$, and $S(x, y) = x + y$. Then we may observe that $S$ is linear, meaning $D_S(a, b) = S$ by (b).
    \item Suppose $P : \RR^2 \to \RR$, and $P(x, y) = xy$. Then\footnote{yes i used binom i'm a lazy fuck}
    \[ D_P(a, b)(x, y) = bx + ay = \begin{pmatrix} b & a \end{pmatrix} \binom{x}{y}. \]
    Then we may write (using $\abs{hk} \leq \sqrt{h^2 + k^2}$)
    \[ \lim_{(h, k) \to 0} \frac{\abs{P(a + h, b + k) - P(a, b) - (bh + ak)}}{\abs{(h, k)}} = \lim_{(h, k) \to 0} \frac{\abs{hk}}{\sqrt{h^2 + k^2}} \leq \sqrt{h^2 + k^2}. \]
\end{enumerate}

\newpage
\begin{simplethm}[Addition, Product, Quotient Rule]
    If $f, g : (U \subset \RR^n) \to \RR^m$ are differentiable at $a \in U$, then
    \begin{enumerate}[label=(\alph*)]
        \item $D(f + g)(a) = Df(a) + Dg(a)$.
        \item If $m = 1$, then $D(fg)(a) = g(a) Df(a) + f(a) Dg(a)$.
        \item If $m = 1$ and $g(x) \neq 0$, then $D\left(\frac{1}{g}\right)(a) = -\frac{1}{g(a)^2} Dg(a)$.
    \end{enumerate}
\end{simplethm}
\noindent
The proofs simply follows from our previous examples.
\begin{enumerate}[label=(\alph*)]
    \item $f + g = S \circ (f + g)$;
    \item $fg = P \circ (f, g)$;
    \item $\frac{1}{g} = \frac{1}{y} \circ g$.
\end{enumerate}

\begin{simplethm}[Jacobian Construction]
    If $f : (U \subset \RR^n) \to \RR^m$ is differentiable at $a$, then $\frac{\partial f_i}{\partial x_j} (a)$ exists for all $1 \leq i \leq n$ and $1 \leq j \leq m$. Then $Df(a)$ is given by the matrix (which we will call the \textit{Jacobian}),
    \[ \begin{pmatrix} \frac{\partial f_1}{\partial x_1} (a) & \dots & \frac{\partial f_1}{\partial x_n} (a) \\ \vdots & \ddots & \vdots \\ \frac{\partial f_m}{\partial x_1} (a) & \dots & \frac{\partial f_m}{\partial x_n} (a) \end{pmatrix} = \begin{pmatrix} \frac{\partial f_i}{\partial x_j} (a) \end{pmatrix}. \]
\end{simplethm}
\noindent To prove this, observe that when $m = 1$, $f : U \to \RR$, and $\frac{\partial f}{\partial x_j}(a) = (f \circ h)'(a_j)$, where $h(x) = (a_1, \dots, a_{j-1}, x, a_{j+1}, \dots)$. Then
\[ \frac{\partial f}{\partial x_j}(a) = (f \circ h)(a_j) = f'(a)\begin{pmatrix} 0 \\ \vdots \\ 1 \\ \vdots \\ 0 \end{pmatrix} j, \]
and so $\frac{\partial f}{\partial x_j}(a)$ is the $j$th component of $Df(a)$.
\medskip\newline
\noindent For general choices of $m$, $f = (f_1, \dots, f_m)$; since each $f_i$ is differentiable at $a$, and $f'(a) = (f_1'(a), \dots, f_m'(a))$, we have that $f_i'(a)$ is the $i$th row of the matrix $f'(a)$. \qed
\medskip\newline
\noindent For example, let $f(x, y) = g(x + y)$. Then $D f(c, d)(x, y) = D_g(c + d)(x + y)$, of which the LHS may be expanded as
\[ (D_g \circ s)(c, d) \circ \underbrace{D_s(c, d)}_{S} (x, y) = D_g(c + d) \circ s(x, y) = D_g(c + d)(x + y) \]
for more clarity.