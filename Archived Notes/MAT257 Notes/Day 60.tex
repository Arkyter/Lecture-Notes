\section{Day 60: Astroid Reparameterization (Mar. 7, 2025)}
We continue the example left off from last class.
\begin{enumerate}[label=(\roman*)]
    \item By first-year calculus, we may observe that an astroid admits two axes of symmetry, where by substituting $x = a \cos^3 \theta$, $y = a \sin^3 \theta$, we obtain the area is
    \[ 4 \int_0^a y \, dx = -4 \int_0^{\pi/2} a \sin^3 \theta \, d(a \cos^3 \theta) = \frac{3\pi a^2}{32}. \]
    \item By change of variables, the parametric region in the first quadrant is given by $x = p \cos^3 \theta$, $y = p \sin^3 \theta$ $(p, \theta) \in [0, a] \times [0, \pi/2]$. Then a quarter of the area of the astroid is given by
    \[ \iint_{[0, a] \times [0, \pi/2]} \abs{\det \frac{\partial(x, y)}{\partial(p, \theta)}} \, dp \, d\theta = 3 \int_0^{\pi/2} \int_0^a p \cos^2 \theta \sin^2 \theta \, dp \, d\theta = \frac{3\pi a^2}{32}. \]
    \item By Stokes' theorem, let $c$ be a $2$-cube parameterizing the astroid. We have that
    \[ \int_c dx \wedge dy = \int_{\partial c} x \, dy = - \int_{\partial c} y \, dx, \]
    since $d(y \, dx) = -dx \wedge dy$ and $d(x \, dy) = dx \wedge dy$. Thus, the above evaluates out to
    \[ \int_0^a (a \cos^3 \theta) = d(a \sin^3 \theta), \]
    which goes to (ii).
\end{enumerate}
We now discuss integration of differential forms on manifolds. Let $M \subset \RR^n$ be a manifold of dimension $k$. Define the following,
\begin{definition}
    A $p$-form on $M$ is an alternating $p$-tensor $\omega : x \mapsto \omega(x) \in \Omega^p(M_x)$, where $M_x$ is the tangent space to $M$ at $x$.
\end{definition}
\begin{definition}
    A vector field on $M$ is a function $F : x \mapsto F(x) \in M_x$ for each $x \in M$.
\end{definition}
\noindent We say that a $p$-form on $M$ is $\SC^r$ if $\varphi^\ast \omega$ is $\SC^r$ for any coordinate chart $\varphi : (W \subset \RR^k) \to (M \subset \RR^n)$. In particular,
\[ (\varphi^\ast \omega)(x)(w_1, \dots, w_p) = \omega(\varphi(x))(\varphi_{\ast x} w_1, \dots, \varphi_{\ast x} w_p). \]
This definition is independent of the choice of coordinate system at a given point of $M$, and we may write any $p$-form as
\[ \omega = \sum_{i_1, \dots, i_p} \omega_{i_1, \dots, i_p} \, dx_{i_1} \wedge \dots \wedge dx_{i_p}, \]
with $\omega_{i_1, \dots, i_p}$ being $\SC^r$.