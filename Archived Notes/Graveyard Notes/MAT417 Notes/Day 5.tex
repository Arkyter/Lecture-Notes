\section{Day 5: Density (Sep. 16, 2025)}
Recall that last time, we discussed that given $m > 0$, we have that
\[ \zeta_m(s) = \prod L(\chi, s) = \prod_{p \nmid m} \left(1 - p^{-f(p)}\right)^{-g(p)}, \]
where the first product is taken over all characters of $(\ZZ/m\ZZ)^\ast$. We have that $f(p)$ denotes the order of $\ol p$, the imag eof $p$, in $(\ZZ/m\ZZ)^
\ast$, and $g(n)$ the number of quotients of $(\ZZ/m\ZZ)$ by the span generated by $\ol p$.
\begin{theorem}
    $\zeta_m(s)$ has a pole of order $1$ at $s = 1$.
\end{theorem}
\begin{corollary}
    $L(\chi, 1) \neq 0$ for all nontrivial characters.
\end{corollary}
\noindent Today, we will use this for the Dirichlet theorem; we will give a more precise formulation of the Dirichlet theorem, and define the notion of density of some set $A \subset \ul P$, where $\ul P$ is the set of all primes.
\begin{lemma}[4.1]
    Given $s \in \RR_{> 1}$, we have that $\sum_p p^{-s} \sim -\log (s-1)$ as $s \to 1$, i.e., the ratio approaches $1$ as $s \to 1$.
\end{lemma}
\noindent Specifically, we have
\[ \sum_p p^{-s} = -\log(s - 1) + O(1). \]
Using the fact that $\zeta(s)$ has a pole of order $1$ at $s = 1$, we have that, for $s \in \RR_{>1}$,
\[ \log \zeta(s) = \sum_{p \in \ul P} -\log(1 - p^{-s}) = \sum_{p,k = 2}^\infty \frac{1}{kp^{ks}} - \frac{1}{p^{ks}} \leq \frac{1}{p^s(p^s - 1)}. \]
It is sufficient to show that $\sum \frac{1}{kp^{ks}}$ remains bounded when $s > 1$, which we readily see from
\[ \sum_{p,k=2}^\infty \frac{1}{kp^{ks}} \leq \sum_p \frac{1}{p^s(p^s - 1)} \leq \sum_p \frac{1}{p(p - 1)} \leq \sum_{n=2}^\infty \frac{1}{n(n-1)} = 1. \]
If $A \subset \ul P$, we say that $A$ has \textit{density} $k \in \RR$ if
\[ \lim_{\substack{s \to 1 \\ s > 1}} \frac{\left(\sum_{p \in A} \frac{1}{p^s}\right)}{-\log (s-1)} = k; \]
clearly, $0 \leq k \leq 1$. We remark that if $k > 0$, then $A$ is an infinite set (all finite $A$ has density zero).
\begin{remark}
    Let $P \subset \NN$ be any infinite subset, and let $A \subset \ul P$. The natural density is defined as
    \[ \lim_{n \to \infty} \frac{\#\{i \in A \mid i \leq n\}}{\#\{i \in \ul P \mid i \leq n\}}, \]
    of which we note this is a stronger notion, since if $A \subset \ul P$ has natural density $k$, then it has density $k$, but the opposite direction is not necessarily true.
\end{remark}
\begin{theorem}
    Let $m > 0$, $\gcd(a, m) = 1$. The set $\ul P_a$ of all primes which are congruent to $a \mod m$ has density $\frac{1}{\varphi(m)}$.
\end{theorem}
\noindent We note that the above is also true for natural density. To prove the theorem, we'll need to know that $L(\chi, 1) \neq 0$ for $\chi \neq 1$. Assuming this is true, we will give the proof as follows; define $f_\chi$,
\[ f_\chi(s) = \sum_{p \nmid m} \frac{\chi(p)}{p^s}, \]
where $s \in \RR_{>1}$ for $s \in \CC$ with real part greater than $1$. To start, observe that $f(1) \sim -\log(s - 1)$ as $s \to 1$; this differs from $\sum_{p \in \ul P} p^{-s}$ by finitely many terms. For $\chi \neq 1$, we have that $f_\chi$ is bounded where $s > 1$; let $g_a(s) = \sum_{p \in P_a} p^{-s}$, and let us claim that
\[ g_a(s) = \frac{1}{\varphi(m)} \sum_\chi \chi(a)\inv f_\chi(s). \]
This yields that $f_\chi(s) \sim -\log(s-1)$, where, if $\chi = 1$ and bounded if $\chi \neq 1$, then we have that
\[ \lim_{s \to 1} \frac{f_\chi(s)}{-\log(s-1)} = \begin{cases} 0 & \chi \neq 1, \\ 1 & \chi = 1. \end{cases} \implies \lim_{s \to 1} \frac{g_a(s)}{-\log(s-1)} = \frac{1}{\varphi(m)}. \]
To fill in the gaps in the above proof outline, observe that
\[ \sum_\chi  \chi(a)^{-1} f_\chi = \sum_{\chi, p \nmid m} \frac{\chi(a)\inv \chi(p)}{p^s}, \quad \sum_{\chi} \chi(a \inv p) = \begin{cases} \varphi(m) & \text{if } a\inv p \equiv 1 \mod m, \\ 0 & \text{otherwise}. \end{cases} \]
More generally, for $G$-finite abelian groups, we have that
\[ \sum_{x \in \wh G} \chi(g) = \begin{cases} \# G & g = 1, \\ 0 & g \neq 1. \end{cases} \]
Moreover, $f_\chi(s) = \sum_{p \nmid m} \frac{\chi(p)}{p^s}$ remains bounded as $s \to 1$, and for $\log L(\chi, s)$, we have that
\[ \log \prod_{p \nmid m} \left(1 - \chi(p) p^{-s}\right)\inv = \sum_{p \nmid m} \sum_{k=1}^\infty \frac{\chi(p)^k}{k p^{ks}} = f_\chi(s) \underbrace{\sum_{p,k \geq 2} \frac{\chi(p)^k}{k p^{ks}}}_{=: B(s)}. \]
In this way,
\[ \abs{B(s)} = \sum_{p,k \geq 2} \frac{1}{kp^{ks}}, \]
which is bounded above as $s \to 1$, so $B(s)$ itself is bounded.