\section{Day 20: Riemann Mapping Theorem, Pt.\ 2, and Harmonic functions (Nov.\ 20, 2025)}
Recall the statement of the Riemann mapping theorem: let $\Omega$ be a proper simply connected open set in $\CC$. For all $z_0 \in \Omega$, there exists a conformal map $F : \Omega \to \DD$ such that $F(z_0) = 0$ and $F'(z_0) = 0$.
\medbreak
\noindent Last class, we showed that there is an injective holomorphic map $F : \Omega \to \DD$ such that $F(z_0) = 0$ and $0 \in \Omega$, which, without loss of generality, we may assume $\Omega$ is an open subset of $\DD$ as well. It remains to show that $\SF = \{f : \Omega \to \DD \mid f \text{ is inj., hol., and } f(0) = 0\}$, for which $s = \sup_{f \in \SF} \abs{f'(0)}$ exists, and there being some $f \in \SF$ such that $\abs{f'(0)} = s$.
\begin{proof}
    By the definition of $s$ and uniform boundedness of $\SF$, there exists some sequence $\{f_n\} \in \SF$ such that $\abs{f_n'(0)} \taking{n \to \infty} s$. By Montel's theorem, there exists a subsequence $\{f_{n_k}\}$ that converges uniformly on compact subsets of $\Omega$. Now, the limit function $f$ is holomorphic on $\Omega$. We will show that $f \in \SF$, i.e., $f(\Omega) \subset \DD$ and $f$ is injective.
    \medbreak
    \noindent By definition, $\abs{f(z)} \leq 1$ on $\Omega$; however, we cannot achieve a maximum on $\Omega$ by the maximum modulus principle, so we indeed have $\abs{f(z)} < 1$, so $f(\Omega) \subset \DD$. Clearly, $f$ is holomorphic so long as $f$ is not constant, which holds since $\abs{f'(0)} = s \geq 1$. We now show that $f$ is injective.
    \begin{proposition}
        Let $\Omega$ be a connected open set, and let $\{f_n\}$ be a sequence of holomorphic injective functions converging to $f$ on $\Omega$. Then $f$ is injective or constant.
    \end{proposition}
    \begin{proof}
        Assume that $f$ is neither injective nor constant; then there exist distinct points $z_1, z_2 \in \Omega$ such that $f(z_1) = f(z_2)$. For $n \in \NN$, define the holomorphic injective functions $g_n : \Omega \to \CC$, given by $g_n(z) = f_n(z) - f_n(z_1)$. Then $z_1$ is the only zero of $g_n$; set $g(z) = f(z) - f(z_1)$; then $g(z)$ admits two zeroes, $z_1$ and $z_2$. Since $f$ is nonconstant, $z_2$ is an isolated zero of $g$, so there is a ball on which $z_2$ is the only zero of $g$; say, $\ol{D_r(z_2)}$ (where $r$ is chosen small enough to exclude $z_1$). By the argument principle, we get that the number of zeroes of $g$ on $D_r(z_2)$ is given by evaluating
        \[ \frac{1}{2\pi i} \int_{\partial D_r(z_2)} \frac{g'(\xi)}{g(\xi)} \, d\xi > 0; \]
        likewise, the number of zeroes of $g_n$ in $D_r(z_2)$ is given by
        \[ \frac{1}{2\pi i} \int_{\partial D_r(z_2)} \frac{g_n'(\xi)}{g_n(\xi)} \, d\xi = 0. \]
        Thus, there is an integer sequence consisting of zeroes entirely, but somehow converges to a number greater than zero; this is clearly a contradiction, so we see that $f$ must be either injective or constant.
    \end{proof}
    \noindent In our context, this proposition tells us that $f$ must be injective, so $f$ is indeed in $\SF$, which means we may follow through with step two and three to conclude the Riemann mapping theorem.
\end{proof}
\noindent We now discuss harmonic functions, i.e., functions with the mean-value property. Let $\Omega$ be a region in $\CC$, and let $u : \Omega \to \RR$. $u$ is said to satisfy the mean value property if, for all $z \in \ol{D_r(z_0)} \subset \Omega$, we have that
\[ u(z_0) = \frac{1}{2\pi} \int_0^{2\pi} u(z_0 + re^{i\theta}) \, d\theta. \]
As shown in problem set two, we've demonstrated that if $u$ is a harmonic function, then $u$ satisfies the mean value property.
\begin{theorem}
    A continuous function $u(z)$ on $\Omega$ satisfying the mean value property is a harmonic function.
\end{theorem}
\begin{theorem}[Poisson's formula]
    Suppose $u(z)$ is harmonic on $\abs{z} < R$ and continuous on $\abs{z} \leq R$. Then\footnote{i believe this is chapter 3, problem 2 in shakarchi.}
    \[ u(a) = \frac{1}{2\pi} \int_{\abs{z} = R} \frac{R^2 - \abs{a}^2}{\abs{z - a}^2} u(z) \, d\theta. \]
\end{theorem}
\begin{proof}
    Consider the linear transformation
    \[ z = S(\xi) = \frac{R(R\xi + A)}{R + \ol a \xi} = R \cdot \frac{(\xi + \frac{a}{R})}{(1 + \frac{\ol a \xi}{R})}, \]
    which we may note looks like a Blaschke factor. Let $u : D_R \to \RR$; we want to show that $u \circ S$ is harmonic by Cauchy--Riemann. By the mean value property of $u \circ S$ at $0$, we have that
    \[ u(a) = u \circ S(0) = \frac{1}{2\pi} \int_{\abs{\xi} = 1} u \circ S(\xi) \, d(\arg \xi); \]
    we will find $d(\arg \xi)$. Observe, that $d(\arg \xi) = \Im \frac{d \xi}{\xi}$, and $\arg = \arctan(y/x)$, so
    \[ d \arctan\left(\frac{y}{x}\right) = - \frac{y}{x^2 + y^2} \, dx + \frac{x}{x^2 + y^2} \, dy. \]
    Alternatively, $\log \xi = \log r + i \arg \xi$, so, locally, we have that $d(\log \xi) = \frac{d\xi}{\xi} = \frac{dr}{r} + d(\arg \xi)$, so $d \arg(\xi) = \Im \left(\frac{d \xi}{\xi}\right)$.
\end{proof}