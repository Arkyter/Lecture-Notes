\section{Day 38: Partition of Unity (Jan. 8, 2025)}
This lecture, we will build up to the idea of a \textit{bump function}. Given $C \subset U \subset \RR^n$, with $C$ compact and $U$ open, there is a $\SC^\infty$ function $f : U \to \RR$ such that $0 \leq f \leq 1$, $f(x) = 1$ for $x \in C$, and $f(x) = 0$ for $x$ outside some compact subset of $U$. The construction is based on\footnote{lost here, idk what is going on lol i got to class late}
\[ \begin{cases} e^{-\frac{1}{x^2}} , &x \neq 0, \\ 0, &x = 0. \end{cases} \]
\begin{enumerate}[label=(\roman*)]
    \item Let $g : \RR \to \RR$ be a $\SC^\infty$ function such that $g(x) > 0$ on $(-1, 1)$ and $g(x) = 0$ outside $(-1, 1)$. Then $g(x)$ can be given by
    \[ g(x) = \begin{cases} e^{-\frac{1}{(x-1)^2}} e^{-\frac{1}{(x+1)^2}}, &x \in (-1, 1), \\ 0, &x \not\in (-1, 1). \end{cases} \]
    \item Given $a \in \RR^n$, $\delta > 0$, there is a $\SC^\infty$ function $g_{a,\delta} : \RR^n \to \RR$  such that $g_{a,\delta}(x) \geq 0$ on the ball $\abs{x - a} < \delta$ and equal to $0$ otherwise. In particular, we may write
    \[ g_{a,\delta}(x) = g\left(\frac{\abs{x - a}^2}{\delta^2}\right). \]
    \item Given $C \subset U \subset \RR^n$, there is a $\SC^\infty$ function $F : \RR^n \to \RR$ such that $F \geq 0$, with $F(x) > 0$ whenever $x \in C$ and $F(x) = 0$ if $x$ is outside some compact subset of $U$. Take $F = \sum_{k} g_{a_k, \delta_k}$.
    \item Given $\eps > 0$, there is a $\SC^\infty$ function $h_\eps : \RR \to \RR$ such that $0 \leq h_\eps \leq 1$, where $h_\eps(x) = 0$ if $x \leq 0$ and $h_\eps(x) = 1$ if $x \geq \eps$. Let $g$ be a $\SC^\infty$ function such that $g(x) \geq 0$ on $(0, \eps)$, and $0$ outside of $(0, \eps)$ with the same construction75 from (ii). Take
    \[ h_\eps(x) = \dfrac{\int_0^x g}{\int_0^\eps g}. \]
\end{enumerate}
We can now write down the definition of the bump function. Let $f(x) = h_\eps \circ F$, where $\eps = \min_{x \in C} F(x)$.
\begin{simplethm}
    Given $A \subset \RR^n$, $\SO$ an open cover of $A$, there is a countable collection $\Phi$ of $\SC^\infty$ functions $\varphi : \RR^n \to \RR$ such that
    \begin{enumerate}[label=(\roman*)]
        \item $0 \leq \varphi \leq 1$,
        \item For all $x \in A$, there is an open neighborhood $V$ of $x$ such that all but finitely many $\varphi \in \Phi$ vanish on $V$.
        \item For all $x \in A$, $\sum_{\varphi \in \Phi} \varphi(x) = 1$; note that the sum is finite as per (ii).
        \item For all $\varphi \in \Psi$, there is $U \in \SO$ such that $\varphi(x) = 0$ outside of some compact subset of $U$.
    \end{enumerate}
\end{simplethm}
\noindent A collection of functions $\Phi$ satisfying the first three properties is called a $\SC^\infty$ \textit{partition of unity} for $A$; if $\Phi$ satisfies (iv), it is said to be \textit{subordinate to $\SO$} with \textit{compact support}. This theorem will be proved next lecture. :3c meow