\section{Day 51: Alternating Tensors and Wedge Product (Feb. 7, 2025)}
Consider the projection $\Alt$ of $\ST^k(V)$ onto $\Omega^k(V)$, where
\[ \Alt T(v_1, \dots, v_k) = \frac{1}{k!} \sum_{\sigma \in S_k} T (\sgn \sigma) (v_{\sigma(1)}, \dots, v_{\sigma(k)}). \]
Let us define the wedge product,
\begin{align*}
    \wedge &: \Omega^k(V) \times \Omega^\ell(V) \to \Omega(k + \ell)(V) \\
    \omega \wedge \eta &= \frac{(k + \ell)!}{k! \ell!} \Alt(\omega \otimes \eta).
\end{align*}
We have that $\wedge$ enjoys a handful of properties, specifically
\begin{align*}
    (\omega_1 + \omega_2) \wedge \eta &= \omega_1 \wedge \eta + \omega_2 \wedge \eta \\
    \omega \wedge (\eta_1 + \eta_2) &= \omega \wedge \eta_1 + \omega \wedge \eta_2 \\
    (a\omega \wedge \eta) &= a (\omega \wedge \eta) \\
    \omega \wedge \eta &= (-1)^{k \ell} \eta \wedge \omega \\
    f^\ast (\omega \wedge \eta) &= f^\ast \omega \wedge f^\ast \eta.
\end{align*}
Additionally, $\wedge$ is associative, i.e. $(\omega \wedge \eta) \wedge \theta = \omega \wedge (\eta \wedge \theta)$, but requires more work to prove.
\begin{simplelemma}[Spivak 4-4]
    The alternating tensor enjoys the following properties,
    \begin{enumerate}[label=(\roman*)]
        \item If $S \in \ST^k(V)$, $T \in \ST^\ell(V)$, and $\Alt S = 0$, then $\Alt (S \otimes T) = \Alt (T \otimes S) = 0$.
        \item If $S \in \ST^k(V)$, $\eta \in \Omega^\ell(V)$, then $\Alt S \wedge \eta = \frac{(k+\ell)!}{k! \ell!} \Alt(S \otimes \eta)$.
    \end{enumerate}
\end{simplelemma}
\noindent The first part of the lemma is left in a future homework assignment. For the second part, we have that
\[ \Alt S \wedge \eta = \frac{(k+\ell)!}{k! \ell!} \Alt(\Alt S \otimes \eta), \]
and so on. (where are we going with this?)
\begin{simplelemma}[Spivak 4-4, part (iii)]
    The alternating tensor also enjoys the property,
    \begin{enumerate}[label=(\roman*)]
        \setcounter{enumi}{2}
        \item Let $\omega \in \Omega^k(V), \eta \in \Omega^\ell(V), \theta \in \Omega^m(V)$, then
        \[ (\omega \wedge \ell) \wedge \theta = \omega \wedge (\ell \wedge \theta) = \frac{(k + \ell + m)!}{k! \ell! m!} \Alt(\omega \otimes \eta \otimes \theta). \]
    \end{enumerate}
\end{simplelemma}
To see that this is true, simply plug in as follows,
\begin{align*}
    (\omega \wedge \ell) \wedge \theta &= \frac{(k + \ell + m)!}{(k + \ell)! m!} \Alt((\omega \wedge \eta) \otimes \theta) \\
    &= \frac{(k + \ell + m)!}{(k + \ell)! m!} \frac{(k + \ell)!}{k! \ell!} \Alt(\omega \otimes \eta \otimes \theta).
\end{align*}
\begin{simpleprop}[Spivak 4-5]
    Let $v_1, \dots, v_n$ be a basis of $V$, and let $\varphi_1, \dots, \varphi_n$ be a dual basis. Then $\varphi_{i1} \wedge \dots \wedge \varphi_{ik}$, over $1 \leq i_1 < \dots < i_k \leq n$, forms a basis of $\Omega^k(V)$. In particular $\dim \Omega^k(V) = \binom{n}{k}$.
\end{simpleprop}
\begin{proof}
    Linear independence is just like before (Spivak 4-1); for the span, let $\omega \in \Omega^k(V)$. Then
    \[ \omega = \sum_{i_1, \dots, i_k = 1}^n a_{i_1, \dots, i_k} \varphi_{i_1} \otimes \dots \otimes \varphi_{i_k}. \]
    Then
    \[ \omega = \Alt \omega = \sum_{i_1, \dots, i_k = 1}^n a_{i_1, \dots, i_k} \Alt(\varphi_{i_1} \otimes \dots \otimes \varphi_{i_k}). \]
    Since each $\Alt(\varphi_{i_1} \otimes \dots \otimes \varphi_{i_k})$ is a constant multiple of $\varphi_{i_1} \wedge \dots \wedge \varphi_{i_k}$, these elements span $\Omega^k(V)$.
\end{proof}
\begin{simpleprop}[Spivak 4-6]
    Let $\omega \in \Omega^n(V)$, $n = \dim V$, and $v_1, \dots, v_n \in V$. Let $w_i = \sum a_{ij} v_j$, where $i = 1, \dots, n$. Then
    \[ \omega(w_1, \dots, w_n) = \det(a_{ij}) \omega(v_1, \dots, v_n), \]
    and so $\omega(v_1, \dots, v_n) = 0$ if $v_1, \dots, v_n$ are linearly dependent.
\end{simpleprop}
\begin{proof}
    Define $\eta \in \ST^n(\RR^n)$ by
    \[ \eta((a_{11}, \dots, a_{1n}), \dots, (a_{n1}, \dots, a_{nn})) = \omega \left( \sum_{j=1}^n a_{ij}v_j, \dots, \sum_{j=1}^n a_{nj} v_j \right). \]
    Then $\eta \in \Omega^n(\RR^n)$, so $\eta = \lambda \cdot \det$ for some $\lambda \in \RR$ and $\lambda = \eta(e_1, \dots, e_n) = \omega(v_1, \dots, v_n)$.
\end{proof}