\section{Day 20: Implicit Function Theorem, Pt. 3; Rank Theorem (Oct. 21, 2024)}
We start with a few remarks on the implicit function theorem.
\begin{enumerate}[label=(\alph*)]
    \item Consider $f(x, y, z) = z^2 - xy^2$;
    \[ \frac{\partial f}{\partial z} = 2z, \hspace{0.2in} \frac{\partial f}{\partial x} = -y^2. \]
    For the first partial, observe that we can solve for $z = g(x, y)$ at any point except where $z = 0$. For the second partial, we see that $-y^2 \neq 0$ as long as $y \neq 0$, meaning we may solve $x = h(y, z)$ near any point $(0, b, 0)$ with $b \neq 0$. Near $(a, 0, 0)$, with $a < 0$, we see that $x$ is defined locally by $y = z = 0$, i.e. $f : (U \subset \RR^3) \to \RR^2$, we have $f(x, y, z) = (y, z)$.
    \item For coordinate changes, recall that if we take $f : \RR^{m + n} \to \RR^n$ and consider $f$ as a function on $(x, y)$, where $x \in \RR^m$ and $y \in \RR^n$, if $f(a, b) = 0$, and if
    \[ \det \frac{\partial f}{\partial y}(a, b) = \det \left( \frac{\partial f_i}{\partial y_j} (a, b) \right) \neq 0, \]
    then there exists a function $F(x, y) = (x, f(x, y))$ with an inverse $H$ near $(a, b)$. We note that $H$ is given by $H(u, v) = (u, h(u, v))$, with $F \circ H = \id$, and $(u, F \circ H(u, v)) = (u, v)$. That is, $(F \circ H)(u, v) = v$.
    \item Suppose $f : \RR^p \to \RR^n$ has rank $n$ at $a$ and $f(a) = 0$, i.e.
    \[ \left(\frac{\partial f_i}{\partial x_j}(a)\right) \]
    has rank $n$ for $1 \leq i \leq n$, $1 \leq j \leq p$. This means we may choose indices $j_1 < \dots < j_n$ such that $\partial_{j_1, \dots, y_n} (f_1, \dots, f_n)(a)$ is invertible. For convenience, let us write $(x_{j_1}, \dots, x_{j_n}) = (y_1, \dots, y_n)$ and $(z_1, \dots, z_{p-n})$ be the other $x$'s. By the implicit function theorem, we can solve for $(y_1, \dots, y_n)$ as a function of $(z_1, \dots, z_{p-n})$. Let $x = P(z, y)$, where $P$ is a linear transformation given by a change of coordinates. Then $(f \circ P)(z, y)$ satisfies $\partial_y (f \circ P) (P(n))$ being invertible, which means there necessarily exists a change of coordinates $(z, y) = H(u, v)$ such that $(f \circ P \circ H)(u, v) = v$. \qed
\end{enumerate}
We now introduce the rank theorem. Let $f : \RR^p \to \RR^n$ be $C^n$, and assume that $f$ has rank $r$ at every point $m$ in a neighborhood of a given point $a$ (so $r \leq p, r \leq n$). In (c) as above, $r = n$; in general, when $r = n$ is rank $r$ at $a$, then it is rank $r$ in some open neighborhood about $a$ by continuity of determinant. Then there exists a coordinate change $x = H(u, v)$ with $v = (v_1, \dots, v_r)$ and $u = (u_1, \dots, u_{p-r})$, and some coordinate change $K$ in the target space such that $(K \circ f \circ H)(u, v) = (v, 0)$. Then we may find indices $i_1 < \dots < i_r$, $j_1 < \dots < j_r$ such that $\partial_{j_1, \dots, j_r}(f_{i_1}, \dots, f_{i_r}) (a)$ has rank $r$. We can assume that $(i_1, \dots, i_r) = (1, \dots, r)$ by permuting the coordinates in the target space; then there is a coordinate change $x = H(u, v)$, with $v = (v_1, \dots, v_r)$ such that $(f_1, \dots, f_r) \circ H(u, v) = v$, i.e.
\[ \underbrace{(f \circ H)}_{= (f_1, \dots, f_r)} (u, v) = (v, (f_{r+1}, \dots, f_n) \circ H(u, v)). \]