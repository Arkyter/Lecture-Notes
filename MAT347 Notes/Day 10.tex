\section{Day 10: Jordan-H\"older Decomposition (Oct. 3, 2025)}
Recall from last class that if $N \nunlhd G$, then $N$, $G/N$ are simple. $\ZZ/n$ is simple if and only if $n$ is prime. The sign function $\sign : S_n \to \{\pm 1\}$ tells you the parity of a permutation, where $\sign \sigma = 1$ if $\sigma$ is even and $-1$ if it is odd. Finally, $A_n = \ker \sign$ is the ``alternating'' group, i.e., the set of even permutations in $S_n$.
\\[8pt]
Per the \href{https://drorbn.net/AcademicPensieve/Classes/25-347-GroupsRingsFields/SimplicityOfAn.pdf}{handout} from last time, we have a proof of why $A_n$ is simple for all but $n = 4$; the proof is just casework bash, so I will not elaborate here. %We have a condensed strategy to approach this. Suppose $\{e\} \neq N \lhd A_n$; we will show that $N$ contains a $3$-cycle $(i, j, k)$.
\\[8pt]
We now move onto Jordan-H\"older.
\begin{definition}
    A Jordan-H\"older decomposition for a group $G$ is a sequence
    \[ G = G_0 \nunrhd G_1 \nunrhd \dots \nunrhd G_n = \{0\} \]
    such that for all $i$, $H_i = G_i/G_{i+1}$ is simple.
\end{definition}
\noindent We give a few motivating examples. Let us consider $\ZZ/n\ZZ$, where $n = p_1 \dots p_k$. Then we have that
\[ \ZZ \nunrhd p_1 \ZZ \nunrhd p_1 p_2 \ZZ \nunrhd \dots \nunrhd n \ZZ. \]
By modding out the entire sequence by $n\ZZ$, we have
\[ \ZZ/n\ZZ \nunrhd \frac{p_1 \ZZ}{n\ZZ} \nunrhd \frac{p_1p_2 \ZZ}{n\ZZ} \nunrhd \dots \nunrhd \frac{n\ZZ}{n\ZZ} = \{e\}; \]
let the elements of the sequence be named $G_0, G_1, \dots, G_k$ in order; observe that we have (where we pick a random index because Dror does that),
\[ H_2 = \frac{G_2}{G_3} = \frac{p_1p_2 \ZZ / n\ZZ}{p_1p_2p_3 \ZZ / n\ZZ} = \frac{p_1p_2 \ZZ}{p_1p_2p_3 \ZZ} \]
by the third isomorphism theorem, and we may factor out $p_1, p_2$ to obtain $\ZZ/p_3\ZZ$, which we know is simple. We may follow the same computation to demonstrate that the above sequence is indeed a Jordan-H\"older decomposition. Note that the decomposition itself is not necessarily unique with respect to the group, however, because as seen above, we may take $p_1, \dots, p_k$ in any order to obtain a sequence with the same property.
\\[8pt]
For another example, consider $G = S_n$ with $n \geq 5$. We have that
\[ G = G_0 \nunrhd G_1 = A_1 \nunrhd G_2 = \{e\}. \]
We have that $H_0 = S_n/A_n = \img(\sign) = \{\pm 1\} = \ZZ/2\ZZ$ which we know is simple, and $H_1 = A_n/\{e\} = A_n$ is simple.
\\[8pt]
For a third example, consider $G = S_4$ (which has $24$ elements). Since $A_4$ is the only non-simple alternating group, we may write
\[ G_0 = S_4 \nunrhd A_4 \nunrhd \ker \phi = \{e, (12)(34), (13)(24), (14)(23)\} \cong \ZZ/2\ZZ \times \ZZ/2\ZZ \nunrhd \{e\}. \]
where $\phi : A_4 \to A_3$ and $H_0 = S_4/A_4 = \ZZ/2\ZZ$, $H_1 = \img \phi = A_3 = \ZZ/3\ZZ$. % https://math.stackexchange.com/a/121495 - reference for klein 4 group here. i'll return to think abt this some more later.