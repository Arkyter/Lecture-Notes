\section{Day 26: PIDs, UFDs, and Greatest Common Divisors (Jan. 9, 2026)}
Recall that a principal ideal domain (PID) states that all ideals are generated by a single element. A unique factorization domain (UFD) admits unique factorization of all nonzero $x$ into $u p_1 \dots p_n$, where $u$ is a unit and $p_1, \dots, p_n$ are primes, up to uniqueness of unit and reordering of primes. A euclidean domain admits a function $e : R \setminus \{0\} \to \NN$ such that $e(ab) \geq e(a) e(b)$ where, for all $a, b \neq 0$, there exists $a = b \cdot q + r$, where either $r = 0$ or $e(r) < e(b)$. It was shown that if $R$ is a PID and $x \neq 0$, then $x$ can be written as a product of irreducibles. It remains to check the following proposition.
\begin{proposition}
    In a PID, irreducible elements are prime.
\end{proposition}
\begin{proof}
    If $x$ is irreducible, $\left<x\right>$ is maximal; indeed, suppose $\left<x\right> \subset J \subset R$ where $J$ is an ideal, then by virtue of being in a PID, there exists $a$ such that $J = \left<a\right>$. Since $x \in J$, there exists $b$ such that $x = a \cdot b$, but $x$ is irreducible, so either $a \in R^\ast$, which is contradictory as now $J = \left<a\right> = R$, or $b \in R^\ast$, meaning $\left<x\right> = \left<a\right>$ as $a = xb\inv$. Thus, $\left<x\right>$ is maximal, so $\left<x\right>$ is prime, and $x$ is prime.
\end{proof}
As an aside, in a UFD, irreducible elements are prime as well.
\begin{definition}
    If $a, b \in R$, $g$ is a \textit{greatest common divisor} of $a$ and $b$ if $g \mid a$, $g \mid b$, and $g' \mid a$, $g' \mid b$ implies $g' \mid g$.
\end{definition}
\begin{claim}
    If $g, g'$ are both gcds of $a, b$, then $g' = ug$ where $u \in R^\ast$, i.e., gcds are unique up to multiplying by a unit.
\end{claim}
\begin{proof}
    By the proposition and the fact that $x$ can be written as a product of irreducibles in PIDs, we see that $g' \mid g$ by symmetry and $g \mid g'$ as well, whence we often call this the ``gcd''.
\end{proof}
In UFDs, gcds are ``easy'', since if $g \mid a$, we can write
\[ a = u \prod_\alpha p_\alpha^{n_\alpha}, g = v \prod_{\alpha} p_\alpha^{k_\alpha} \iff k_a \leq n_\alpha \]
for all $\alpha$. Note that all but finitely many of the $n_\alpha$ and $k_\alpha$ are zero. In this manner, if $b = w \prod_\alpha p_\alpha^{m_\alpha}$, we obtain
\[ \gcd(a, b) = \prod p_\alpha^{\min(m_\alpha, n_\alpha)}. \]
As an example, $\gcd(30, 18) = 6$ and $\gcd(x^2 - 2x + 1, x^2 - 1)$ in $\QQ[x]$ is $x - 1$ by factorizing the former as $(x - 1)^2$ and the latter as $(x - 1)(x + 1)$.
\begin{theorem}
    If $R$ is a PID and $a, b \in R$, then $\left<a, b\right> = \left<\gcd(a, b)\right>$.
\end{theorem}
\begin{proof}
    By PID, there exists $y$ such that $\left<y\right> = \left<a, b\right>$, implying $y \mid a$ and $y \mid b$, so $y \mid g$, whence $g \in \left<y\right>$, which gives the $\supset$ direction. We will discuss the other direction next time.
\end{proof}
\begin{corollary}
    In a PID, $a, b \in R$, $g = \gcd(a, b)$, there exists $s, t$ such that $g = sa + tb$.
\end{corollary}