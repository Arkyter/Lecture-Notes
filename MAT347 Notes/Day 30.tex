\section{Day 30: Modules as a Semiring (Jan.\ 23, 2026)}
Let $R$ be a (commutative, since we assume so in this class) ring. Then the $R$-modules form a semiring under $\oplus, \otimes$ as the binary operations and $0, R$ as the additive and multiplicative identities. We claim the following,
\begin{proposition}
    The modules form a ring as defined above with the following properties,
    \begin{enumerate}[(i)]
        \item $R$ is an identity for $\otimes$, 
        \item $0$ is an identity for $\oplus$,
        \item $\oplus, \otimes$ are associative.
        \item $\otimes$ distributes over $\oplus$
        \item $\oplus$ is commutative.
    \end{enumerate}
\end{proposition}
Note that $\oplus$ does not have additive inverses, so the modules do not form a ring, but a semiring instead. We will leave (ii) and (v) as an exercise. For property (i), we want to show that $R \otimes M \cong M \cong M \otimes R$. We will show the left side, since commutativity would yield the right. To start, we use the universal property of tensor products on $R \otimes M$.
\[ \begin{tikzcd} R \times M \arrow[r, hook, "\iota"] \arrow[dr, "f"] & R \otimes M \arrow[d, "\exists! \wt f"] \\ & N \end{tikzcd} \]
The above diagram shows that, for all bilinear $f \colon R \times M \to N$, we have that $f(r, m) = f(1, rm)$. By considering the map $R \times M \to M$ by $(r, m) \mapsto rm$ and define $\wt f(m) = f(1, m)$, we get
\[ \begin{tikzcd}[column sep=2cm] R \times M \arrow[r, "\ell\colon (r{,}m) \mapsto rm"] \arrow[rd, "f"] & M \arrow[d, "\wt f(m)=(1{,}m)"] \\ & N \end{tikzcd} \]
By uniqueness of $\wt f$, we get that $\wt f(rm) = f(1, rm) = f(r, m)$ as desired. Thus, $(M, \ell)$ satisfies the universal property, meaning $M \cong R \otimes_R M$.
\\[8pt]
We now discuss (iii). To see $\otimes$ is associative, i.e., $A \otimes (B \otimes C) \cong (A \otimes B) \otimes C$, observe that we may write
\[ A \times (B \times C) \leftrightarrow (A \times B) \times C \leftrightarrow (A \otimes B) \times C \leftrightarrow (A \otimes B) \otimes C \]
Reading the above from left to right, we see that for each $a$, we get a bilinear map $\ell_a$, and we have that the following diagram commutes,
\[ \begin{tikzcd} B \times C \arrow[r] \arrow[dr, "\ell_a"] & B \otimes C \arrow[d, "\exists! \wt \ell_a"] \\ & (A \otimes B) \otimes C \end{tikzcd} \]
where $\ell_a(b, c) = (a \otimes b) \otimes c$, whence $\wt \ell_a$ is the unique map given by
\[ \wt \ell_a \left(\sum b_i \otimes c_j\right) = \sum \wt \ell_a \left(b_i \otimes c_j\right). \]
$\wt \ell$ is linear in $a$, and it is the restriction of a trilinear map. In particular, we have the bilinear map
\[ \begin{tikzcd} A \times (B \otimes C) \arrow[r] \arrow[rd, "(a{,} b \otimes c) \mapsto \wt \ell_a(b \otimes c)"'] & A \otimes (B \otimes C) \arrow[d, dashed] \\ & (A \otimes B) \otimes C \end{tikzcd} \]
and the induced map (dashed arrow) is a unique module homomorphism. As an aside, if $f \colon M_1 \to M_2$ is a $R$-module homomorphism, and $N$ is an $R$-module, then there is a map
\begin{align*}
    N \otimes M_1 &\to N \otimes M_2, \\
    n \otimes m_1 &\mapsto n \otimes f(m_1).
\end{align*}
More generally, if $g \colon N_1 \to N_2$ as well, then we also get $N_1 \otimes M_1 \to N_2 \otimes M_2$ defined by $n \otimes m \mapsto g(n) \otimes f(m)$.
\\[8pt]
We now discuss (iv): distributivity, i.e., $A \otimes (B \oplus C) \cong (A \otimes B) \oplus (A \otimes C)$. Observe the diagram
\[ \begin{tikzcd} A \times (B \oplus C) \arrow[r] \arrow[rd] & A \otimes (B \oplus C) \arrow[d, dashed, "f"] \\ & (A \otimes B) \oplus (A \otimes C) \end{tikzcd} \]
As in the previous proof, we use the map $(a, (b, c)) \mapsto (a \otimes b, a \otimes c)$. This is bilienar, and so by the universal property of tensor products, the above diagram gives us a map $f \colon A \otimes (B \oplus C) \to (A \otimes B) \oplus (A \otimes C)$ as a module homomorphism. Next, observe the following diagram.
\[ \begin{tikzcd}
    A \times B \arrow[r] \arrow[dr] & A \otimes B \arrow[d, "\text{univ prop}"] \arrow[dr, "\text{proj}"] & \\
    & A \otimes (B \oplus C) & (A \otimes B) \oplus (A \otimes C) \arrow[l] \\
    A \times C \arrow[r] \arrow[ur] & A \otimes C \arrow[u, "\text{univ prop}"'] \arrow[ur, "\text{proj}"'] &
\end{tikzcd} \]
The property of direct sums (from homework 11) tells us that we get the other homomorphism, from which it follows that $g$ must have that $g(a \otimes b, 0) = a \otimes (b, 0)$, and $g(0, a \otimes c) = a \otimes (0, c)$. Indeed, we obtain
\[ f \circ g(a \otimes b, a \otimes c) = f(a \otimes (b, 0) + a \otimes (0, c)) = f(a \otimes (b, c)) = (a \otimes b, a \otimes c). \]
Now, we also have $g \circ f(a \otimes (b, c)) = g \circ (a \otimes b, a \otimes c)$, which is given by $a \otimes (b, 0) + a \otimes (0, c) = a \otimes (b, c)$. Thus, $g = f\inv$ demonstrates the isomorphism. Note that the map we've chosen here is a module homomorhpism $A \otimes B \to A \otimes (B \oplus C)$, and so it must be the correct one by uniqueness.