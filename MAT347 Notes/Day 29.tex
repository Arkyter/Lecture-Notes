\section{Day 29: Tensor Products (Jan.\ 21, 2026)}
Given $M, N$ modules, we can form a new module $M \oplus N$ as seen previously. We will now do tensor products.
\begin{definition}
    A tensor product of $M, N$ is a module $M \otimes_R N$ and an $R$-bilinear map $i : M \times N \to M \otimes_R N$ such that, for all $R$-modules $P$ and $p : M \times N \to P$, there exists $\tilde p : M \otimes_R N \to P$ such that $p = \hat p \circ i$, i.e., the following diagram commutes.
    \[ \begin{tikzcd} M \times N \arrow[d,"i"] \arrow[dr,"p"] \\ M \otimes_R N \arrow[r,"\hat p"'] & P \end{tikzcd} \]
\end{definition}
When is $i$ injective and when is it surjective? Observe that $i(v, 0) = i(v, 0 \cdot 0) + 0 \cdot i(v, 0) = 0 = i(0, 0)$, so it is never injective; surjective is tricky.
\begin{lemma}
    $\spn(\Img(i)) = M \otimes_R N$.
\end{lemma}
\begin{proof}
    Observe that the diagram
    \[ \begin{tikzcd} M \times N \arrow[d,"i"] \arrow[dr,"{p(v,w)=0}"] \\ M \otimes N \arrow[r,"\hat p"'] & M \otimes N/\spn(\Img(i)) \end{tikzcd} \]
    commutes, where $\pi\colon M \otimes N \to M \otimes N/\spn(\Img(i))$ has $\pi \circ i = 0$ by definition, whence $\pi \circ i = p$, and $\pi = \hat p = 0$ by uniqueness. In this manner, we see it must be that $\spn(\Img(i)) = M \otimes_R N$.
\end{proof}
As an example of a tensor product, consider $M = \RR^m$ and $N = \RR^n$; then $i(v, w) = v^t w$, and we have $i\colon M \times N \to M_{m\times n}(\RR)$.\footnote{this is suspicious}
\begin{theorem}[Existence and uniqueness]
    Tensor products exist and are unique up to a unique isomorphism.
\end{theorem}
\begin{proof}
    For existence, let $G$ be a free abelian group generated by $\{m \cdot n \mid m \in M, n \in N\}$; this is not a module yet; we need an action of $R$ on $G$ such that
    \[ (r + s)v = rv + sv, \quad r(v_1 + v_2) = rv_1 + rv_2, \quad r(sv) = (rs)v, \quad 1v = v. \]
    Let $H < G$ be genearted by things like
\end{proof}