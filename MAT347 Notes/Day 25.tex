\section{Day 25: Rings like \texorpdfstring{$\ZZ$}{Z} (Jan. 7, 2025)}
Recall that, for commutative rings, $R/I$ is a field if and only if $I$ is maximal, and $R/I$ is a domain (admits no zero divisors) if and only if $I$ is prime, i.e., $ab \in I$ implies $a \in I$ or $b \in I$.
\\[8pt]
Now, we assume that our rings are not commutative domains. If $p$ is prime, then $p \mid ab$ implies $p \mid a$ or $p \mid b$; equivalently, $\left<p\right>$ is prime. If $x$ is irreducible, then $x$ is \textit{not} $0$, not a unit, and $x = ab$ implies $a \in R^\ast$ or $b \in R^\ast$. In general, prime implies irreducible but the converse is not necessarily true.
\\[8pt]
Recall that the properties of the integers are as follows,
\begin{enumerate}[(i)]
    \item Unique factorization, i.e., any integer $n$ can be written as $p_1 \dots p_n$. Equivalently, if a ring $R$ possesses this property, we call it a unique factorization domain (UFD).
    \item $\left<4, 6\right> = \left<2\right>$, i.e., even numbers. This property means $R$ is a principal ideal domain (PID).
    \item There exists $\abs{\cdot}$, i.e., $347 = 37 \cdot 9 + 14$, where $\abs{14} > \abs{37}$. We say that $R$ is an Euclidean domain here. 
\end{enumerate}
\begin{definition}
    $R$ is a \textit{unique factorization domain} if any $x \neq 0$ can be written as $x = u p_1 p_2 \dots p_k$, where $u \in R^\ast$ and each $p_i$ is prime.
\end{definition}
\begin{definition}
    $R$ is a \textit{principal ideal domain} if every ideal in it is ``principal'', meaning generated by a single element.
\end{definition}
\begin{definition}
    $R$ is a \textit{Euclidean domain} if there exists $e : R \setminus \{0\} \to \NN$ such that $e(ab) \geq e(a)$ and, for all $a, b \neq 0$, there exists $q, r$ such that $a = b \cdot q + r$ and $r = 0$ with $e(r) < e(b)$.
\end{definition}
\begin{theorem}
    Every Euclidean domain is a principal ideal domain, and every principal ideal domain is a unique factorization domain.
\end{theorem}
As an example, $\QQ[t]$ is a Euclidean domain and a PID, but $\ZZ[t]$ and $\QQ[s, t]$ is a UFD but not a PID.
\begin{theorem}
    If $R$ is a UFD and $x = u p_1 \dots p_n$ and $x = v q_1 \dots q_m$, where $u, v$ are units and each $p_i, q_i$ are primes, then $(p_1, \dots, p_n)$ and $(q_1, \dots, q_m)$ are the same up to units and a permutation.
\end{theorem}
\begin{proof}
    We konw that $p_1 \mid q_1 \dots q_m$, so by definition of $p_1$ being prime, $p_1 \mid q_j$ for some $j$. Without loss of generality, $p_1 \mid q_1$, so $p_1 \sim q_1$. We may induct on this process to obtain our desired result.
\end{proof}
\begin{claim}
    $p_1 \mid q_1$ and $q_1$ being prime means $p_1 = q_1$ up to multiplication by a unit.
\end{claim}
\begin{proof}
    Indeed, $q_1 = p_1 \cdot a$, but $q_1$ is irreducible, so either $p_1$ is a unit or $a$ is a unit (with only the latter being possible).
\end{proof}
\begin{theorem*}
    Euclidean domains are PIDs.
\end{theorem*}
We start with some examples.
\begin{enumerate}[(i)]
    \item $\ZZ$. Let $e : \ZZ \setminus \{0\} \to \NN$ and $e(k) = \abs{k}$.
    \item $\QQ[t]$ or $F[t]$, where $F$ is a field. LEt $f = \sum_{i=1}^n a_i t^i$, where $a_n \neq 0$ and $e(f) = \deg f = n$.
\end{enumerate}
\begin{claim}
    $e$ is a Euclidean domain.
\end{claim}
This is clear from $e(gf) \geq e(f)$ assuming $f, g \neq 0$; the latter is clear from long divison. We now prove the theorem.
\begin{proof}
    Suppose $R$ is a Euclidean domain with norm $e$. Suppose $I \subset R$ is an ideal. Let $x \neq 0$ be an element of $I$ with the least possible $e$. We claim that $I = \left<x\right>$; indeed, suppose $y \in I$, write $y = q \cdot x + r$, where $e(r) < e(x)$ or $r = 0$. Then $r = y - qx \in I$, implying $r = 0$ or $y \in \left<x\right>$.
\end{proof}
\begin{theorem*}
    PIDs are UFDs.
\end{theorem*}
\begin{lemma}
    If $R$ is a PID, then $R$ is \textit{Noetherian}, meaning you cannot find a sequence $I_1, I_2, \dots$ of ideals in $R$ such that $I_1 \subsetneq I_2 \subsetneq I_3 \dots$.
\end{lemma}
\begin{proof}
    Suppose such a sequence existed. Then take $J$ to be their union; it is clear that $J$ is an ideal, as $R$ is a PID, meaning $J = \left<x\right>$ for some $x$, so $x \in \bigcup I_i$, so there exists $N$ such that $x \in I_n \subset I_{n+1} \dots$. Thus, $I_n \supset \left<x\right> = J$ and $I_{n+1} \supset \left<x\right> = J$, whence $I_n = I_{n+1} = \dots = J$.
\end{proof}
\begin{lemma}
    If $R$ is a PID, and $x \in R$ with $x \neq 0$, then $x$ is either a unit or a product of irreducibles.
\end{lemma}
\begin{proof}
    Suppose not; then there exists $x \in R$ that is not a unit, not irreducible, and not a product of irreducibles. Since $x$ is not irreducible, $x = x_1 \cdot x_1'$ such that $x_1, x_1' \notin R^\ast$; at leasto ne of $x_1$ and $x_1'$ are not a product of irreducibles. Without loss of generality, $x_1$ isn't a product of irreducibiles; continue in the same fashion, and we get two seqeunces; $x_k$ and $x_k'$ such that $x_k, x_k' \notin R^\ast$, where $x_k = x_{k+1} x_{(k+1)'}$. Now, consider that
    \[ \left<x_1\right> \subsetneq \left<x_2\right> \subsetneq \dots. \]
    This contradicts the previous lemma, meaning that for some index (we will pick that index to be $2$ for now) $x_2 \in \left<x_1\right>$, then $x_2 = ax_1 = a \cdot x_2 \cdot x_2'$, implying $x_2'$ is a unit.
\end{proof}