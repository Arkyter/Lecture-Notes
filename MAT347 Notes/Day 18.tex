\section{Day 18: Finitely Generated Abelian Groups (Nov.\ 7, 2025)}
If $M$ is a finitely generated abelian group, we want to show that there exists $r, p_i, s_i$ where $1 \leq i \leq k$ such that $p_i$ is prime with\footnote{i'm just going to link \href{https://crypto.stanford.edu/pbc/notes/group/fgabelian.html}{this}}
\[ M \cong \ZZ^r \times \prod_{i=1}^k \ZZ/p_i^{s_i}. \]
Furthermore, $r$ is determined uniquely, as are the primes and their powers, up to permutation. Recall that if a group $G$ is finitely generated, we have that $G = \left<g_1, \dots, g_n\right>$ (write $X$ to be the set $\{g_1, \dots, g_n\}$); we can consider the free group $F(X)$, and consider the ``interpretation'' group action $F(X) \to G$ as taking a word and applying the group operations in $G$ to the word.\footnote{since the structure of the free group differs from that of $G$; think of it as a restriction on the words, perhaps.} Let us provide a proof sketch of the above claim.\footnote{note that this $G$ and $X$ are unrelated to what's to come later.}
\begin{proof}[Proof sketch]
    We will prove the claim by ``Gaussian elimination on something''. We start with the finite case for intuition; for any matrix $A \in M_{m \times n}(\ZZ)$, let us construct a finitely generated abelian group $M_A$ (then we shall show all finitely generated abelian groups are of this form) as follows,
    \[ A \mapsto (\phi_A : \ZZ^n \to \ZZ^m) \mapsto M_A := \frac{\ZZ^m}{\Img \phi_A}. \]
    For some examples of this process, observe that
    \begin{enumerate}[(i)]
        \item If $A = 0$, then $\phi_A : \ZZ \to \ZZ$, so $M_{(0)} = \ZZ/0 = \ZZ$.
        \item If $A = I_n$, then $\phi_A : \ZZ^n \to \ZZ^n$, where $M_{I_n} = \ZZ^n/\ZZ^n = \{e\}$. \item If $A \in M_{1 \times 1}(\ZZ)$, write $A = \ell$; then $\phi_A : \ZZ \to \ZZ$ is simply given by multiplication by $\ell$, so $M_\ell = \ZZ/\ell$.
    \end{enumerate}
    This can be generalized $A \in M_{G \times X}(\ZZ)$, where $G$ is a finite set, and $X$ not necessarily finite to also yield finitely generated abelian groups $M_A$. In particular, the operation would go as
    \[ A \mapsto (\phi_A : \ZZ^X \to \ZZ^G) \mapsto M_A = \frac{\ZZ^G}{\Img \phi_A}. \]
    We may note that $\ZZ^X$ is the set of $f : X \to \ZZ$ with finite support (similarly, all functions $G \to \ZZ$ already have finite support), which is a group under addition, and $M_A$ is finitely generated because $\ZZ^G$ is finitely generated.
\end{proof}
\begin{claim}
    Every finitely generated abelian group $M$ has a finite $G$, a not necessarily finite $X$, and $A \in M_{G \times X}$ such that $M \cong M_A$.\footnote{we omit the $\ZZ$ from $M_{G \times X}$ from now on for conciseness. damn dror is getting footnoted hard}
\end{claim}
\begin{proof}
    %Take $M$ as in the claim; by assumption, $M$ is generated by the finite set $G$. Since $M$ is abelian, all words in $G$ can be ordered, i.e., $g_1^{a_1} \dots g_n^{a_n}$, where $a_i \in \ZZ$, or a linear combination of the elements. Thus, $Z^G \curvearrowright M$ under $\pi$ is onto $\{\pi : (a_1, \dots, a_n)^\top \mapsto g_1^{a_1} \dots g_n^{a_n}\}$. By the first isomorphism theorem, we have that
    %\[ \pi(\ZZ^G) \cong \ZZ^G/\ker \pi \implies M \cong \ZZ^G / \ker \pi, \]
    %which is abelian as it is a subgroup of $\ZZ^G$. Pick $X$ to be the generators of $\ker \pi$ (which are not necessarily finite); then
    %\[ \ZZ^X \stackrel{\pi}{\mapsto} \ker(\pi) \xhookrightarrow{} \ZZ^G \taking{\pi} M \tag{$\phi : \ZZ^X \to \ZZ^G$}. \]
    %Now, $\phi : \ZZ^X \to \ZZ^G$ is represented by some matrix $A$, whose elements are elements of $\ZZ^X$.\footnote{whatever the hell is going on here}
    Since $M$ is a finitely generated abelian group, let $M = \left<g_1, \dots, g_n\right>$, for which we will let $G = \{g_1, \dots, g_n\}$. Since $M$ is abelian, we may reorder any word of $G$ as $g_1^{a_1} \dots g_n^{a_n}$; define $\pi : \ZZ^G \to M$ by $\pi((a_1, \dots, a_n)^\top) = g_1^{a_1} \dots g_n^{a_n}$; by the first isomorphism theorem, we obtain
    \[ \pi(\ZZ^G) \cong \ZZ^G / \ker \pi \implies M \cong \ZZ^G / \ker \pi, \]
    for which we know $\ker \pi$ is abelian as it is a subgroup of $\ZZ^G$. Let $X$ be the generators of $\ker \pi$ (note that this is not necessarily finite), and observe that the interpretation $\tilde \pi$ on $\ZZ^X$ sends to $\ker \pi$, for which we have
    \[ \begin{tikzcd}
        \ZZ^X \arrow[r, "\tilde \pi"] \arrow[rr, bend right=20, "\phi"'] & \ker \pi \arrow[r, hook] & \ZZ^G \arrow[r, "\pi"] & M
    \end{tikzcd} \]
    As seen in the diagram above, we may define $\phi : \ZZ^X \to \ZZ^G$ and represent it by some matrix
    \[ A = \begin{pNiceArray}{c|c|c} & & \\ \cdots & \phi(x_\alpha) & \cdots \\ & & \end{pNiceArray}, \]
    where each column is given by $\phi(x_\alpha)$ for each $x_\alpha \in X$ (which can be thought of as each $e_\alpha \in \ZZ^X$). \textit{Note: the intuition for $A$ is that each column describes which words vanish in $M$.}
\end{proof}