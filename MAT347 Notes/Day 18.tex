\section{Day 18: Finitely Generated Abelian Groups (Nov.\ 7, 2025)}
If $M$ is a finitely generated abelian group, we want to show that there exists $r, p_i, s_i$ where $1 \leq i \leq k$ such that $p_i$ is prime with\footnote{i'm just going to link \href{https://crypto.stanford.edu/pbc/notes/group/fgabelian.html}{this}}
\[ M \cong \ZZ^r \times \prod_{i=1}^k \ZZ/p_i^{s_i}. \]
Furthermore, $r$ is determined uniquely, as are the $p_i, s_i$'s up to permutation. Recall that if a group $G$ is finitely generated, we have that $G = \left<g_1, \dots, g_n\right>$ (write $X$ to be the set $\{g_1, \dots, g_n\}$); we can consider the free group $F(X)$, and consider the ``interpretation'' group action $F(X) \curvearrowright G$ as taking a word and applying the multiplications in $G$ to the word. Let us provide a proof sketch of the above claim.
\begin{proof}[Proof sketch]
    ``Gaussian elimination on something''; consider a matrix $A \in M_{m \times n}(\ZZ)$ and construct a finitely generated abelian group $m_A$ (then we shall show all finitely generated abelian groups are of this form) as follows,
    \[ A \mapsto (\phi_A : \ZZ^n \to \ZZ^m) \mapsto M_a := \frac{\ZZ^m}{\Img \phi_A}. \]
    Here are some examples. If $A = 0$, then $\phi_A : \ZZ \to \ZZ$, so $M_{(0)} = \ZZ/0 = \ZZ$. If $A = I$, then $\phi_A : \ZZ^n \to \ZZ^n$, where $M_{(I)} = \ZZ^n/\ZZ^n = \{e\}$. If $A = \ell$, then $\phi_A : \ZZ \to \ZZ$ (given by multiplication by $\ell$), so $M_\ell = \ZZ/\ell$.
    \\[8pt]
    This can be generalized with the same map to $A \in M_{G \times X}(\ZZ)$, where $G$ is finite, $X$ not necessarily finite, to yield finitely generated abelian groups $M_A$. In particular, the operation would go as
    \[ A \mapsto (\phi_A : \ZZ^X \to \ZZ^G) \mapsto M_A = \frac{\ZZ^G}{\Img \phi_A}. \]
    We may note that $\ZZ^X$ is the set of $f : X \to \ZZ$ with finite support, which is a group under addition, and $M_A$ is finitely generated because $\ZZ^G$ is finitely generated.
\end{proof}
\begin{claim}
    Every finitely generated abelian group $M$ has a finite $G$, an $X$, and $A \in M_{G \times X}$ such that $M \cong M_A$.
\end{claim}
\begin{proof}
    Take $M$ as in the claim; by assumption, $M$ is generated by the finite set $G$. Since $M$ is abelian, all words in $G$ can be ordered, i.e., $g_1^{a_1} \dots g_n^{a_n}$, where $a_i \in \ZZ$, or a linear combination of the elements. Thus, $Z^G \curvearrowright M$ under $\pi$ is onto $\{\pi : (a_1, \dots, a_n)^\top \mapsto g_1^{a_1} \dots g_n^{a_n}\}$. By the first isomorphism theorem, we have that
    \[ \pi(\ZZ^G) \cong \ZZ^G/\ker \pi \implies M \cong \ZZ^G / \ker \pi, \]
    which is abelian as it is a subgroup of $\ZZ^G$. Pick $X$ to be the generators of $\ker \pi$ (which are not necessarily finite); then
    \[ \ZZ^X \stackrel{\pi}{\mapsto} \ker(\pi) \xhookrightarrow{} \ZZ^G \taking{\pi} M \tag{$\phi : \ZZ^X \to \ZZ^G$}. \]
    Now, $\phi : \ZZ^X \to \ZZ^G$ is represented by some matrix $A$, whose elements are elements of $\ZZ^X$.\footnote{whatever the hell is going on here}
\end{proof}