\section{Day 3: Rubik's Cube, Pt. 3 (Sep. 10, 2025)}
Before we return to the discussion on the Rubik's cube, we have another property of inverses to discuss;
\begin{theorem}
    Let $a \in G$. Then $(a\inv)\inv = a$.
\end{theorem}
\begin{proof}
    $(a\inv)\inv = (a\inv)\inv \cdot (a\inv \cdot a) = a$.
\end{proof}
\noindent The point of the twist is that we want to fill every box of our table that can be filled by the group; assuming that the twist hits everything, we would be able to work nicely with the group by just unfurling each permutation progressively. As an example, given $(\dot{z_1}, \dot{z_2}, \dots)$, we wish to find the index $k$ such that $\dot{z_k} = 1$. We may then apply $\sigma_{1,k}^{-1}$ to $(\dot{z_1}, \dot{z_2}, \dots)$ to obtain $(1, \dots)$, on which we may then recursively proceed. Inventing this gives us $(\dot{z_1}, \dot{z_2}, \dots)$ in terms of the generators.
\begin{lemma}
    Every $\sigma_{i, j} \in G$.
\end{lemma}
\begin{proof}
    We fed generators or elements of the table into the table, but each feed only performs group opreations, which means inductively, we are done here.
\end{proof}
\begin{lemma}
    Any $\sigma \in S_n$ fed into the table is a monotone product of elements of $T$. We have that $\sigma = \sigma_{1, j_1} \cdot \sigma_{2, \cdot{j_2}} \cdot \dots \cdot \sigma_{n, j_n}$, where our $\sigma_{i, j_i}$s are drawn from the table, and the box in the index $(i, j_i)$ is nonempty.
\end{lemma}
\begin{proof}
    There are three possibilities;
    \begin{enumerate}[(i)]
        \item If $\sigma = e$, then it's just $\sigma_{1,1} \sigma_{2,2} \sigma_{3,3} \dots$.
        \item If $\sigma$ is in the table, suppose its $\sigma_{i,j}$; then $\sigma = \sigma_{1,1} \dots \sigma_{i,j} \dots \sigma_{n,n}$. 
        \item If $\sigma$ is neither of these, then suppose $\sigma$ has pivot $i$, $\sigma(i) = j$, and $\sigma_{i,j}$ is full; then we just feed $\sigma' = \sigma_{i,j}\inv \sigma$. In other words, $\sigma = \sigma_{i,j} \sigma'$, and since you can only repeat this finitely many times, this is eventually $\sigma e \sigma_{1,1} \dots \sigma_{i,j} \dots \sigma_{n,j_n}$. \qedhere
    \end{enumerate}
\end{proof}
\noindent The before holds for $\sigma \in S_n$ fed into the table, but we don't necessarily have that the table $T$ generates the group just yet. If we feed in a generator $g_i$, we have that $g_i$ is in $\left<T\right>$, meaning that by feeding our generators, we indeed have that $\left<T\right> = G$. Therefore, feeding products of elements is to get everything in $G$ as a monotone product.
\begin{lemma}
    Two monotone products are equal if and only if they are the same.
\end{lemma}
\begin{proof}
    If they are the same, they are equal. Suppose that $\sigma_{1,j_1} \dots \sigma_{n,j_n} = \sigma_{1,j_1'} \dots \sigma_{n,j_n'}$. Then
    \[ \sigma_{i,j_1} \dots \sigma_{n,j_n} = (\sigma_{i,j_1}\inv \sigma_{1,j_1'}) \sigma_{2,j_2'} \dots \sigma_{n,j_n'}, \]
    meaning that
    \[ \sigma = \sigma_{n,j_n}\inv (\sigma_{n-1,j_{n-1}}\inv (\dots(\sigma_{1,j_1}\inv \sigma_{1,j_1})\dots)\sigma_{n-1,j_{n-1}'})\sigma_{n,j_n'} = e, \]
    but then we have
\end{proof}