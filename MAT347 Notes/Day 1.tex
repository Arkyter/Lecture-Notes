\section{Day 1: Rubik's Cube}
The first semester of this class will be taught by Dror Bar-Natan instead of Joe Repka. Since this was a last minute change, the Quercus, tutorials, textbook, homework policy, etc. are all unknown for now (until the rest of the week probably).
\\[8pt]
This will be today's \href{https://www.math.toronto.edu/~drorbn/Talks/Cambridge-1301/}{handout}. Let $G = \left<g_1, \dots, g_\alpha\right>$, i.e., the group generated by $g_1, \dots, g_\alpha$, be a subgroup of $S_n$, with $n = O(100)$. To understand $G$, let us start by computing $\abs{G}$. \textit{insert long digression about Rubik's cubes that can be read elsewhere}.
%\\[8pt]
\begin{definition}
    A \textit{group} is a set $G$ along with a binary multiplication $m : G \times G \to G$ usually written as $(g_1, g_2) \mapsto g_1 \cdot g_2 = m(g_1, g_2)$ such that
    \begin{enumerate}[(i)]
        \item $m$ is associative, i.e., for all $g_1, g_2, g_3 \in G$, we have that $(g_1 g_2) g_3 = g_1 (g_2 g_3)$,
        \item $m$ has an identity, i.e., there exists some $e \in G$ such that $g \cdot e = e \cdot g = g$ for all $g \in G$,
        \item $m$ has an inverse, i.e., for all $g \in G$, there exists some $h \in G$ such that $g \cdot h = e = h \cdot g$,
    \end{enumerate}
\end{definition}
\noindent We present a few examples of groups for intuition.
\begin{enumerate}[(a)]
    \item $(\ZZ, m = +)$, $(\QQ, +)$, $(\RR, +)$, $(\CC, +)$, $(F, +)$ are naturally all groups. We also have that $(2 \ZZ, +)$ is a group, even though it is not a field, because it does not admit inverses.
    \item $(\QQ \setminus \{0\}, \times)$ has identity given by $1$ and naturally admits inverses because all reciprocals are contained within $\QQ \setminus \{0\}$ itself. We commonly write the rationals without zero as $\QQ^\times$.
    \item If $n \in \ZZ_{\geq 0}$, then let $S_n := \{\sigma : \ul{n} \to \ul{n} \mid \sigma \text{ is bijective} \}$, where we define $\ul{n} = \{1, \dots, n\}$.\footnote{angry yapping incoming i am so used to seeing $[n]$ when i saw that on the board i was like, watefak!!!} Let the group operation on $S_n$ be given by composition. Here, Dror goes into a big digression on how composition should be written, and he suggests the following,\footnote{also, plus one angry footnote for using $/\!\!/$ as a composition symbol}
    \[ \sigma \cdot \mu = \mu \circ \sigma = \sigma /\!\!/ \mu. \]
    Indeed, $S_n$ is a group, where its identity element $e$ is given by the identity function on $\ul{n}$. We have that $\abs{S_n} = n!$.
    \\[8pt]
    As a substantive example, consider $S_2 = \{[1, 2], [2, 1]\}$, where $[1, 2]$ represents the identity function and $[2, 1]$ represents the function mapping $1$ to $2$ and $2$ to $1$. Then we obtain the following possible compositions,
    \begin{align*}
        [1, 2] [1, 2] &= [1, 2], \\
        [1, 2] [2, 1] &= [2, 1], \\
        [2, 1] [1, 2] &= [2, 1], \\
        [2, 1] [2, 1] &= [1, 2].
    \end{align*}
    As for $S_3$, we have that $S_3$ contains $6$ functions, comprised of all the possible permutations possible on $\{1, 2, 3\}$. One such composition is given as follows,
    \[ [1, 3, 2] [2, 1, 3] = [2, 3, 1], \qquad [2, 1, 3] [1, 3, 2] = [3, 1, 2], \]
    which confirms that $S_3$ is indeed not abelian (i.e., non-commutative).
    \\[8pt]
    In the opposite direction, $S_1$ consists of an identity function only; clearly, $\abs{S_1} = 1! = 1$. $S_0$ is the set of all permutations on $\ul{0}$, which is clearly the empty set, meaning the ``empty function'' on the empty set is the only function in $S_0$; similarly, $\abs{S_0} = 0! = 1$.
    \item There are $24$ rotational symmetries of a cube.
    \item The orthogonal transformations $o(3) = \{A \in M_{3 \times 3}(\RR) \mid A \cdot A^\top = I\}$ form a group.
\end{enumerate}
\begin{simplethm}
    The identity element of a group is unique. If $G$ is a group and $e, e'$ are both identity elements, then for all $g \in G$, we have that $eg = ge = g$ and $e'g = ge' = g$, and $e = e'$.
\end{simplethm}
\begin{proof}
    Observe that $e' = e' \cdot e = e$.
\end{proof}
\begin{simplethm}
    The inverse of an element in a group is unique. Let $G$ be a group and $g \in G$; if $h, h'$ satisfy $gh = hg = e = gh' = h'g$, then $h = h'$.
\end{simplethm}
\begin{proof}
    Observe that $h' = h' \cdot e = h' (gh) = (h' g) h = eh = h$.
\end{proof}
\noindent From here on, the inverse of $g$ will be denoted $g^{-1}$, i.e., $g^{-1}$ is the unique inverse of $g$.
\begin{simplethm}
    If $ac = bc$ in a group then $a = b$.
\end{simplethm}
\begin{proof}
    Given that $ac = bc$, we have $ac c^{-1} = bc c^{-1}$, implying $a = b$.
\end{proof}
\begin{simplethm}
    $(ab)^{-1} = a^{-1} b^{-1}$.
\end{simplethm}
\begin{proof}
    Observe that $(ab) (b^{-1}a^{-1}) = abb^{-1}a^{-1} = aea^{-1} = aa^{-1} = e$.
\end{proof}
\begin{definition}
    A subset $H \subset G$ of a group $G$ is called a subgroup if $H$ is closed under multiplication, $e \in H$, and admits inverses (i.e., $H$ is a group itself with the multiplication operation from $G$). We write $H < G$.
\end{definition}
\noindent As an example, $(2 \ZZ, +) < (\ZZ, +) < (\QQ, +), (\RR, +), (\CC, +)$. The Rubik's cube group is also a subgroup of $S_{54}$.