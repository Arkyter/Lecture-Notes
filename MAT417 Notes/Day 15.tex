\section{Day 15: (Oct. 21, 2025)}
Last class, we discussed that, given a number field $K$ and a ring of integers $\SO_K$, some examples of a non-principal ideal in $\SO_K$. Let $R$ be a ring, and let $\ka \subset R$ be an ideal; we say that it is principal if there exists some $\alpha \in R$ such that $\ka = \{x\alpha \mid x \in R\}$. If $\ka = (\alpha_1, \dots, \alpha_n)$, we have that $\ka = \{\sum_i x_i \alpha_i \mid x_i \in R\}$.
\\[8pt]
As an example, let $K = \QQ(\sqrt{-5})$. We have that $\SO_K = \ZZ[\sqrt{-5}]$, where we have the ideal $\ka = (2, 1 + \sqrt{-5})$, for which we observe is non-principal. Indeed, if $\ka = (\alpha)$, then $2 = \alpha x$ and $1 + \sqrt{-5} = \alpha y$; since $4 = N(2) = N(\alpha)N(x)$ and $6 = N(1 + \sqrt{-5}) = N(\alpha)N(y)$, we have that $N(\alpha)$ divides both $4$ and $6$, where $N(\alpha) \neq \pm 1$. Supposing $N(\alpha) = \pm 2$, we also obtain a contradiction since
\[ \alpha = a + b \sqrt{-5} \implies N(\alpha) = a^2 + 5b^2 \neq 2 \]
for any $a, b \in \ZZ$.
\\[8pt]
To show that $(2, 1 + \sqrt{-5}) \neq \SO_K$, it is enough to show that $\SO_K / 2 \SO_K$ is a ring over $\FF_2$ of dimension $2$. Indeed, $\SO_K / 2 \SO_K = \FF_2[x] / (x^2 - 1)$, and $1 + \sqrt{-5} \in \SO_K$ generates a proper ideal $1 + x$ of $\FF_2[x] / (x^2 - 1)$.\footnote{what are we cooking here}
\\[8pt]
There are two problems to deal with. \begin{parlist} \item there is no unique decomposition into irreducibles, and \item not all ideals are principal. \end{parlist}
\begin{lemma}
    Let $\ka \subset \SO_K$ be a proper ideal. Then $N(\ka) = \# (\SO_K / \ka)$ is finite.
\end{lemma}
\begin{proof}
    It is enough to prove for principal ideals that $\SO_K \cong \ZZ^n$; (he then erased the board without elaborating).
\end{proof}
\begin{definition}
    We define the Dedekind $\zeta$-function,
    \[ \zeta_K(s) = \sum_{\ka} N(\ka)^{-s}. \]
    If $K = \QQ$, this is exactly the Riemann $\zeta$-function.
\end{definition}
\noindent We want to show that $\zeta_K(s)$ has properites similar to $\zeta_\QQ(s)$. We also want to discuss its applications, i.e., if $m$ is an integer, then
\[ K = K_m = \QQ(e^{2\pi i/m}) = \QQ(\mu_m), \]
where $\mu_m$ is a root of $1$ of order $m$. $\zeta_K(s)$ will be essentially the intersection of $L(\chi, s)$, where $\chi : (\ZZ/m\ZZ)^\ast \to \CC^\ast$. Let $R$ be a ring, and let $\kp \subset R$ be an ideal. We say that $\kp$ is prime if $R/\kp$ has no zero divisors, i.e., for all $\alpha, \beta \in R$ such that $\alpha \beta \in \kp$, either $\alpha$ or $\beta$ is in $\kp$.
\\[8pt]
Suppose $\ka_!, \ka_2$ are two ideals of $R$. Then
\[ \ka_1 \cdot \ka_2 = \left\{\sum_i \alpha_i \beta_i \mid \alpha_i \in \ka_1, \beta_i \in \ka_2\right\} = \{\alpha \beta \mid \alpha \in \ka_1, \beta \in \ka_2\}, \]
which is again an ideal.
\begin{theorem}
    For $K$ a number field and $\ka \in \SO_K$, there exists a unique decomposition (up to permutation) $\ka = p_1 \dots p_n$ where each $p_i$ is prime.
\end{theorem}
\noindent We say that a fractional ideal $\ka$ of $\SO_K$ is a finitely generated $\SO_K$-submodule of $K$.
\begin{lemma}
    If $\ka \subset \SO_K$ is an ideal, then $\ka$ is fractionally generated over $\SO_K$ if and only if $\SO_K$ is Noetherian.
\end{lemma}
\noindent As an example, consider $K = \QQ$ and let $\ka \subset \QQ$ be a fractional ideal. Then $\ka = (\alpha)$, where $\alpha \in \QQ$; we see that the fractional ideals of $\QQ$ are $\QQ_{\geq 0}$ with non-zero fractional ideals being isomorphic to $\QQ_{>0}$. Under multiplication, $\QQ_{>0}$ is an abelian group $\QQ^\ast / \pm 1$. We can define a product of fractional ideals $\ka_1 \cdot \ka_2$, with the product generated by all $\alpha_i \beta_j$.
\begin{theorem}
    Nonzero fractional ideals in $K$ form an abelian group with respect to multiplication.
\end{theorem}
\noindent In this way, let $F$ be the group of fractional ideals, and observe that the proper principal fractional ideals are $K^\ast / \SO_K^\ast$. Let $U(K)$ denote this quotient.
\begin{theorem}
    $U(K)$ is finite.
\end{theorem}
\noindent We say that the class number of $K$ is given by $\# U(K)$. If $U(K)  = 1$, then all fractional ideals are principal, and so all ideals of $\SO_K$ are principal. As an example, consider the quadratic field $\QQ(\sqrt{-d})$ where $d > 0$ is square free. We have that the class number is $1$ if $d = 1, 2, 3, 7, 11, 19, 43, 67, 163$. It is conjectured that there are infinitely many such $d$ such that the class number is $1$.