\section{Day 1: Recap of Preliminaries (Jan.\ 6, 2026)}
Today's class can be followed more precisely on \S1.2 to \S1.4 of our textbook by \href{https://link.springer.com/book/10.1007/978-3-031-25409-3}{Gross and Meinrenken}. The slogan of this class is that a manifold is something that locally looks like $\RR^n$. Specifically, an $n$-manifold can be covered $n$-dimensional charts $(U \subset M) \to \RR^n$, with our main motivating example being solutions sets to equations. Recall the implicit function theorem,
\begin{theorem}
    Given a smooth function $f : \RR^{n+1} \to \RR$, consider the solution set $f(x_1, \dots, x_{n+1}) = 0$ and a point $p \in \RR^n$ such that $\nabla f(p) \neq 0$; then, for $(x_1, \dots, x_{n+1})$ in said solution set near $p$, we can represent solutions as $(x_1, \dots, x_n, g(x_1, \dots, x_n))$, where $g$ is also a smooth function.
\end{theorem}
In particular, if $0$ is a regular value\footnote{note to self: what's a regular value?} of $f$, then we can cover $\{x \mid f(x) = 0\}$ by graphs/charts. We present some examples;
\begin{enumerate}[(i)]
    \item Let $f(x, y) = xy$; then $\ker f$ is precisely the $x$ and $y$ axes, which is not a manifold, because it does not look like $\RR^n$ (for any $n$) near the origin.
    \item Let $f(x, y) = y - x^{2/3}$; then $\ker f$ can be graphed in desmos as $y = x^{2/3}$, which is not a smooth manifold because of its behavior at $0$.
    \item The $n$-sphere $S^n = \{x \in \RR^n \mid x_1^2 + \dots + x_{n+1}^2 = 1\}$ can be regarded as the level set of the $\ell^2$-norm, for which $S^0 = \{\pm 1\} \subset \RR$, $S^1$ is a circle, $S^2$ is the usual sphere. Note that we may use the stereographic projection as seen in complex analysis, to view $S^3$ (and any of the previous or subsequence $S^n$) as $\RR^3 \cup \{\infty\}$.
    \item The $2$-dimensional torus $T^2$ is the surface of revolution obtained from a circle of radius $r$ and $R$ about an axis of revolution. It can be regarded as a level set by writing
    \[ T^2 = \{(x, y, z) \in \RR^3 \mid (\sqrt{x^2 + y^2} - R)^2 + z^2 = r\}. \]
    \item The M\"obius strip can't be a part of a level set (at a regular value) because level sets are orientable (2-sided), while the strip is not.
    \item The Klein bottle is also not orientable; it is closed (doesn't have a boundary), and doesn't embed into $\RR^3$. It can be immersed into $\RR^3$, i.e., locally embedded but not globally, as seen in the textbook.
\end{enumerate}
\begin{theorem}[Whitney Embedding Theorem]
    Every $n$-manifold has an embedding in $\RR^{2n}$.
\end{theorem}
In this class, we prefer to deal with intrinsic descriptions of manifolds rather than extrinsic ones; a good motivation is given on p.7 in the textbook with respect to our $2$-torus.
\\[8pt]
Now, consider $M$ to be the rotations of a ball. We call this a configuration spcae; i.e., its points are a way of configuring another object. How do we put coordinates on a piece of this space? To start, we wish to describe $M$; one possible way is by considering the $3 \times 3$ orthonormal matrices, i.e., $\opname{SO}(3)$; another way is to first designate a point on the unit sphere as the north pole $N$ (of which there are two degrees of freedom in this choice), then choosing where the vector $(1, 0, 0)$ at $N$ is mapped to (of which we have one degree of freedom).
\\[8pt]
In this manner, we may regard $M$ as a $3$-dimensional manifold and define coordinates for points near the identity.
\\[8pt]
A second example is given by considering linkages, which are collections of line segments and joints (p.8-10); suppose we have four segments given by $n_1, \dots, n_4$, to be regarded as vectors in $\RR^2$; without additional constraints imposed, we see that this can be regarded as $8$-dimensional. In imposing the conditions on \textit{making} $n_1, \dots, n_4$ a linkage, we see that $\norm{n_i}$ for each $i = 1, \dots, 4$ is fixed (whence $4$ less dimensions), $n_1 + n_2 + n_3 + n_4 = 0$ (whence one less), and $n_1$ is fixed (whence one less again), so such a linkage can be regarded as a $2$ dimensional manifold with coordinates $\theta$ and $\varphi$.
\\[8pt]
Finally, denote $\RP^2$ the real projective plane, given by lines passing through the origin in $\RR^3$; equivalently, we may regard this as the pairs of antipodal $\{x, -x\}$ points in $S^2$.
\begin{fact}
    ``Closed'' (equivalently, compact) surfaces are easy to enumerate.
\end{fact}
Fedya then said something about orientability, but that's left in the textbook and easier to read there.
\\[8pt]
We now wish to define manifolds. For a first attempt, we want to say that a smooth manifold is a set $M$ covered by a set of smooth charts $\varphi_i : (U_i \subset M) \to \RR^n$, such that each $p \in M$ is covered by some $U_i$. However, we run into a problem; what does ``smooth'' mean? $f : U \to \RR^n$ recall that a function is said to be smooth if it has partial derivatives of all orders. A function $f : U \to V$ is a diffeomorphism if it's a smooth bijectino and its inverse is also smooth.
\begin{definition}
    A coordinate chart is an injective map $\varphi : U \to \RR^n$ with open image for some $U \subset M$.
\end{definition}
In particular, we say that two charts $\varphi : U \to \RR^n$ and $\psi : V \to \RR^n$ are compatible if the transition $\psi \circ \varphi\inv : \varphi(U \cap V) \to \psi(U \cap V)$ is a diffeomorphism. Clearly, if $U \cap V = \emptyset$, then said maps are compatible; note, however, that compatibility is not an equivalence relation, since, while it is symmetric and reflexive, it is not necessarily transitive.