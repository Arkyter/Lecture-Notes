\section{Day 8: Morse Functions}
A function from \( \mathbb{R}^{n} \to \mathbb{R} \) is morse if at critical points its hessian is nondegenerate. 

\begin{lemma}[Morse Lemma]
If \( f \colon \mathbb{R}^{n} \to \mathbb{R} \) and \( x_{0} \) is a nondegenerate critical point of \( f \) then there are local coordinates \( x_{1}', \ldots, x_{n}' \) in a neighborhood of \( x_{0} \) such that \( f \) in these coordinates is a quadratic form. Namely,
\[ f(x_{1}', \ldots, x_{n}') = f(x_{0}) + \sum_{i=1}^{n} \sum_{j=1}^{n} \frac{\partial^{2} f}{\partial x_{i} \partial x_{j}}(x_{0}) x_{i}'x_{j}'. \]
\end{lemma}
We will not prove this right now. Notice that in our original coordinates,
\[ f(x) = f(x_{0}) + df_{x_{0}}(x - x_{0}) + \sum_{i=1}^{n} \sum_{j=1}^{n} \frac{\partial^{2} f}{\partial x_{i} \partial x_{j}}(x_{0}) x_{i}x_{j} + O(||x - x_{0}||^{3}) \]
so the Morse lemma comes to proving we can eliminate this remainder. Also by definition, the linear term cancels out.

If \( f \colon M \to \mathbb{R} \) and \( M \) is a closed manifold, then suppose that \( f(M) = [0, 1] \). Then by considering the sets \( f^{-1}([0, x]) \) we can learn a lot about the topology of \( M \) and about the nondegenerate critical values as well. The signature (the negative eigenvalues of the Hessian) tell you about homology groups and Betti numbers and all sorts of fancy stuff.

Suppose we have a different coordinate system \( \psi \). Define \( f^{\psi} \) to be \( f \circ \psi \) (I don't know why he used this notation). Then we have that if \( x_{0} \) is a critical point of \( f \) then \( \psi^{-1}(x_{0}) \) is a critical point of \( f^{\psi} \). What we want to show is that if \( f \) has nondegenerate Hessian, then so does \( f^{\psi} \).

Notice now that
\[ \frac{\partial f^{\psi}}{\partial x_{i}} = \sum_{\alpha=1}^{n} \frac{\partial f}{\partial x_{\alpha}} \frac{\partial \psi_{\alpha}}{\partial x_{i}}. \]
As such,
\[ \frac{\partial^{2} f^{\psi}}{\partial x_{i} \partial x_{j}} = \sum_{\alpha=1}^{n}  \frac{\partial f}{\partial x_{\alpha}} \frac{\partial^{2} \psi_{\alpha}}{\partial x_{i} \partial x_{j}} + \sum_{\alpha, \beta = 1}^{n} \frac{\partial^{2} f}{\partial x_{\alpha} \partial x_{\beta}} \frac{\partial \psi_{\alpha}}{\partial x_{i}} \frac{\partial \psi_{\beta}}{\partial x_{j}} = D^{T} \psi D^{2} f D \psi.  \]
where the first sum is zero because we are considering a critical point of \( f^{\psi} \).

Suppose now that \( f \colon U \to \mathbb{R} \) which is not necessarily morse. We want to show that there is a linear function such that perturbing \( f \) by this linear function makes it morse.
\begin{lemma}
	The measure of vectors such that \( f_{a} \colon U \subseteq \mathbb{R}^{n}\to \mathbb{R} \) given by
	\[ f_{a}(x) = \left\langle x, a \right\rangle + f(x) \]
	is \textit{not} morse is \( 0 \).
\end{lemma}
\begin{proof}
	\( d \left( f_{a} \right) = df + a \) so the critical points of \( f_{a} \) are points where \( df(x_{0}) = -a \). The upshot is that \( f_{a} \) is morse if for all \( x \), \( \nabla f(x) = -a \implies d( \nabla f)(x) \) is nondegenerate. This is equivalent to saying that \( -a \) is a regular value of \( \nabla f \), which is true for all but a measure zero set of \( a \) by Sard's theorem.
\end{proof}

\begin{theorem}
	If \( f \colon M^{n} \to \mathbb{R} \) where \( M^{n} \subseteq \mathbb{R}^{N} \), then we have that \( f_{a} = f + a \cdot x \) is morse for almost all \( a \in \mathbb{R}^{N} \).
\end{theorem}
\begin{proof}
	Let \( x \in M^{n} \) and consider our coordinate functions \( x_{1}, \ldots, x_{N} \) on \( \mathbb{R}^{N} \). Then we can suppose (without loss of generality) that the first \( x_{1}, \ldots, x_{n} \) form a coordinate system for some neighborhood \( V \) of \( x \). Then it follows that for any tuple \( \left( c_{n+1}, \ldots, c_{N} \right) \), we can define the function
	\[ f_{(0, c_{n+1}, \ldots, c_{N})} = f + c_{n+1} x_{n+1} \cdots + c_{N} x_{N}. \]
	Then for almost all \( b \in \mathbb{R}^{n} \),
	\[ f_{(b, c)} = f_{(0, c)} + b_{1} x_{1} + \cdots + b_{k} x_{k} \]
	is going to be Morse, so applying Fubini's yields that on this neighborhood of \( x \) the function \( f \) is Morse. Now, if we notice that we can cover \( M^{n} \) with countably many neighborhoods (since \( f \) is second-countable it is in particular Lindel\"of) we can extend this from a statement about a neighborhood of one point to a statement about all of \( M^{n} \). Isn't second-countability nice?
\end{proof}
Next class we'll prove Whitney's embedding theorem for non-compact manifolds. This will be tedious for me.
