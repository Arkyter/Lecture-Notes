\section{Day 7: Homotopies and Stability Theorem}
Let's suppose that we have some \( f_{0} \colon M \to N \). Then we define a homotopy to be a smooth function \( F \colon M \times [0, 1] \to N \) such that
\[ F(x, 0) = f_{0}(x). \]
Sometimes we write \( f_{t}(x) = F(x, t) \). As Nabutovsky said suggestively, \( F \) is a \( 1 \)-parametric family of \( t \).

Let's suppose that \( f_{0} \) has some nice property. Then we want to show that for sufficiently small times \( F \) preserves this property. This leads us to the following monster of a theorem:
\begin{theorem}
	Suppose that \( X \) is a compact manifold, \( Y \) is a manifold, and \( f_{0} \colon X \to Y \) is smooth. Let \( F \colon X \times [0, 1] \to Y \)  be a smooth homotopy with \( F(x, 0) = f_{0}(x) \). Then if \( f_{0} \) is
	\begin{enumerate}
	
		\item A local diffeomorphism
		\item An immersion
		\item A submersion
		\item Transversal to some \( Z \subseteq Y \)
		\item An embedding
		\item A diffeomorphism
	
	\end{enumerate}
	then for some sufficiently small \( \varepsilon \), all of the maps \( f_{t} \) share this property for \( t < \varepsilon \). In other words, these properties are preserved by homotopy for small enough times.
\end{theorem}
\begin{proof}
	\begin{enumerate}
	
		\item Really a local diffeomorphism is just an immersion and a submersion. But if \( \dim X = \dim Y \) then by the rank-nullity theorem of linear algebra, we need only show our map is an immersion. As such, it suffices to prove 2.
		\item Let's show it locally first. To do this, notice that for each point \( x \) there exists a collection of indices \( I \) such that the minor consisting of only the rows \( I \) of the matrix \( df_{0}(x) \) is invertible. Let \( \widetilde{df_{0}}(x)_{I} \) be this minor. Then notice that
			\[ \det \widetilde{df_{0}}(x)_{I} > 0 \]
			As such, since the determinant (and the Jacobian matrix) are continuous, there exists some open neighborhood \( U_{x} \subseteq X \times [0, 1] \) with \( (x, 0) \in U_{x} \) such that for all \( (x_{0}, t) \in U_{x} \),
			\[ \det \widetilde{df_{t}}(x)_{I} > \frac{1}{2} \det \widetilde{df_{0}}(x)_{I}.  \]
			It follows that on this neighborhood, \( f_{\bullet} \) is an immersion. As such, there's an open neighborhood of \( X \times \left\{ 0 \right\} \) on which \( f_{\bullet} \) is an immersion, so the compactness of \( X \) yields that there exists some \( \varepsilon > 0 \) such that \( f_{\bullet} \) is an immersion on \( X \times [0, \varepsilon) \).
		\item Submersions are proven in the exact same way as immersions. There's a minor of the Jacobian which has nonzero determinant, etc.
		\item Recall from the section on transversality that this condition is equivalent to \( g \circ f_{0} \) being an immersion on \( f_{0}^{-1}(Z) \). Thus it reduces to immersion.
		\item We know that immersions and submersions are preserved, so we need to preserve injectivity. Suppose that we have \( \left\{ (x_{n}, y_{n}, t_{n}) \right\} \) such that
				\[ x_{n} \ne y_{n} \quad f_{t_{n}}(x_{n}) = f_{t_{n}}(y_{n}) \quad \lim_{n \to \infty}t_{n} = 0. \]
				Without loss of generality (by the compactness of \( X \)) we can suppose that our \( x_{n} \) and our \( y_{n} \) both converge. As such, suppose that
				\[ x = \lim_{n \to \infty} x_{n} \ne \lim_{n \to \infty} y_{n} = y \]
				Then since \( F(x, 0) \ne F(y, 0) \), it follows by the continuity of \( F \) that there exists open neighborhoods \( U_{x} \), \( U_{y} \) of \( \left( x, 0 \right) \) and \( (y, 0) \) such that \( F(U_{x}) \cap F(U_{y}) = \varnothing \). But we have that for some \( n \in \mathbb{N} \), \( (x_{n}, t_{n}) \in U_{x} \) and \( (y_{n}, t_{n}) \in U_{y} \) and thus
				\[ f_{t_{n}}(x_{n}) \ne f_{t_{n}}(y_{n}), \]
				which is a contradiction. It follows that \( \lim_{n \to \infty} x_{n} = \lim_{n \to \infty} y_{n} \).

				But then set \( G(x, t) = (F(x, t), t) \). Then we know that \( DG(x, 0) \) is invertible as it is of the form
				\[ \begin{bNiceMatrix}DF(x, 0) & * \\ 0 & 1\end{bNiceMatrix}  \]
				where \( DF(x, 0) \) has full rank, and this is meant to be a block matrix.

				It follows that we can apply the immersion theorem to get that on some neighborhood of \( (x, 0) \), \( G \) is an injection. But since \( t_{n} \to 0 \) and \( x_{n} \to 0 \) and \( y_{n} \to 0 \), it follows that for any large enough \( n \),
				\[ G(x_{n}, t_{n}) = G(y_{n}, t_{n}) \implies (x_{n}, t_{n}) = (y_{n}, t_{n}) \]
				which is a contradiction. From here it's just another compactness argument to make the choice of \( \varepsilon \) uniform across all points.

		\item Every embedding is diffeomorphic to its image, so since we know that embeddings are preserved, we need only show that the image is the same for sufficiently small times. Nabutovsky devised the world's worst proof of this fact so I'll present mine which he said was good.

			Suppose that \( (U, \varphi) \) is a parameterization of \( X^{n} \) and \( (V, \psi) \) is a parameterization of \( Y^{n} \). Then if \( f_{0}(\varphi(x_{0})) = \psi(y_{0}) \), we want to show that there is some function \( h \colon [0, \varepsilon) \to U \) such that for all \( t \in [0, \varepsilon) \),
			\[ \psi^{-1} \circ f_{t} \circ \varphi(h(t)) = y_{0}. \]
			Let
			\[ G(x, t) = \psi^{-1} \circ f_{t} \circ \varphi(x) - y_{0} \]
			Since \( F \) is a smooth homotopy, we have that \( G \) is smooth, so since \( G(x_{0}, 0) = 0 \), and \( G \) has full rank in its first \( n \) columns (this is equivalent to \( f_{0} \) being a submersion), by the implicit function theorem, there exists precise some \( h \) as above. 

			Now, it's just another compactness argument where we show that we can find a common \( \varepsilon \) for every point of \( X \). This is because we have that our map \( h \) varies continuously on \( x_{0} \).
	\end{enumerate}
\end{proof}

Sard's theorem has one more application. Specifically, Whitney's embedding theorem.
\begin{theorem}[Whitney's Embedding Theorem]
	Suppose that \( X^{n} \) is compact and \( X^{n} \subseteq \mathbb{R}^{N} \). Then there exists an embedding from \( X \) to \( \mathbb{R}^{2n+1} \).
\end{theorem}
\begin{remark}
	We don't need compactness, but then we need some way to show our map is proper, which is more difficult. As such, we will assume compactness \textit{for now}. Also, 

	Why do we bother proving Whitney's? Well a priori we could have that \( N \) is huge. As such, it's nice to have an upper bound on the dimension of the space. Is \( 2n+1 \) a tight upper bound? No, in fact it's never the optimal dimension. We can show that all manifolds can be embedded into \( \mathbb{R}^{2n} \), but it's much harder to do. Even in this case, we can show that sometimes manifolds can be embedded into spaces of lower dimension, but this is still an open problem in differential topology (what is the optimal dimension we can embed into?
\end{remark}
\begin{proof}
	We're gonna show that so long as \( N > 2n+1 \) there exists some vector \( a \in \mathbb{R}^{N} \) with \( a \ne 0 \) such that the orthogonal projection \( \pi_{a} \colon \mathbb{R}^{N} \to \left\langle a \right\rangle^{\perp} \) is an embedding for \( X^{n} \) into \( \mathbb{R}^{N-1} \). 

	In general, we cannot just pick any vector because it might not be injective and our map \( \pi_{a} \) might not be an immersion. So, let us identify when our \( a \) is bad. Define the map \( h \colon X \times X \times \mathbb{R} \to \mathbb{R}^{N} \) such that
	\[ h(x, y, t) = t(x - y). \]
	Then we have that if \( a \in \operatorname{im} h \), this implies that there exists \( x \in X \), \( y \in Y \), \( t \in \mathbb{R} \) such that \( tx = ty + a \), and thus \( \pi_{a}(tx) = \pi_{a}(ty) \).

	Notice that if \( N > 2 \dim X + 1 \), then we have that every point of \( \operatorname{im} h \) will be critical. But then by mini-Sard's, the image of \( h \) is going to be meagre in \( \mathbb{R}^{N} \), so really \textit{most} choices of \( a \) will work.

	But this still might not be an embedding. This is because we need to ensure that our map is also an immersion. To do this, we need to look at the tangent bundles. The tangent bundle \( TM \) of a manifold is the collection of all its tangent spaces, and the tangent bundle of an \( n \)-manifold will have dimension \( 2n \). Another way to put this is that
	\[ TM = \left\{ (x, v) \colon x \in M, v \in T_{x}M \cong \mathbb{R}^{n}. \right\} \]

	If \( \varphi \) is a parameterization of our manifold \( X \), then we are in trouble when \( d \varphi_{x_{0}}(v) = ta \) for some nonzero \( v \) and some \( t \in \mathbb{R} \). This is because in this case
	\[ d\left( \pi_{a} \circ \varphi \right)_{x_{0}}(v) = d \left( \pi_{a} \right)_{\varphi(x_{0})}( d \varphi_{x_{0}}(v)) = \pi_{a}(ta) = 0, \]
	so our map \( \pi_{a} \) wouldn't be an immersion. But to avoid this, notice that the mapping \( g \colon TM \to \mathbb{R}^{N} \) by
	\[ g(x, v) = d \varphi_{x}(v) \]
	will have measure zero image as long as \( N > 2n \). Thus, since we can cover \( M \) via countably many coordinate charts, the bad points will form a measure zero subset of \( \mathbb{R}^{N} \).
\end{proof}
\begin{remark}
	We can show that there exists an \textit{immersion} into \( \mathbb{R}^{2n} \) by the above argument. Sure, when we project from \( \mathbb{R}^{2n+1} \) to \( \mathbb{R}^{2n} \) we may lose injectivity, but that's okay for an immersion. There exists an embedding too, but you just have to trust him on that one.
\end{remark}
Notice that this is slightly disappointing, because we went through the trouble of proving Sard's theorem and yet so far we've only used the easier version of it. But it's not the worst thing in the world. You'll live.
