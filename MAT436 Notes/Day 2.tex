\section{Day 2: (Sep. 5, 2025)}
\begin{exercise}
    How many diagonals are there in a parallelepiped in an arbitrary vector space?
\end{exercise}
\noindent When we started talking about the projection theorem, which is one of the properties of Hilbert spaces, i.e., for every closed linear subspace, there is a complementary linear subspace where the intersection is $0$ and the sum is the whole space.
\\[8pt]
Any Hilbert space has an orthonormal basis.
\begin{exercise}
    Hilbert spaces can be vector spaces. Given a Hilbert space basis, if it is not finite dimensional, then this cannot be a vector space basis.
\end{exercise}
\noindent Let $H$ be a hilbert space, and let $K$ be a closed subspace of $H$. We call $K^\perp$ the \textit{orthogonal complement}, consisting of the vectors perpendicular to all vectors in $K$. Is it true that $H = K + K^\perp$? Note that even though $K \cap K^\perp = \{0\}$, we do not use the direct sum here; we refer to $+$ as the internal direct sum. $\oplus$ is reserved for external direct sums, if we wish to create a third Hilbert space from two separate Hilbert spaces.
\\[8pt]
Let $H$ be a Hilbert space, and let $S$ be a closed and convex subset of $H$, and let $x \in H$. There exists a unique closest point of $S$ to $x$. It can be proven using the parallelogram law. Similarly, given $H = K + K^\perp$ and $x \in H$, there exists a closest point to $x$ in $K$, which we shall call $y$. Then we have that $x - y \in K^\perp$.
\\[8pt]
Try to find something related to the lectures for the next class. Specifically, index theory, etc, finding stuff on Wikipedia is fine.